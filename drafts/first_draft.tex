\documentclass[11pt,a4paper,oneside]{article}
\usepackage[polish]{babel}
\usepackage{polski}

\usepackage[utf8]{inputenc}
\usepackage{DejaVuSerifCondensed}
\usepackage[T1]{fontenc}
\usepackage{indentfirst}
\frenchspacing

\setlength{\parindent}{0em}
\setlength{\parskip}{2.5em}
\linespread{1.15}
\setlength{\emergencystretch}{3em}  % prevent overfull lines

\pagestyle{empty}

\begin{document}
\textbf{nil. słuchowisko}\\
pierwszy fragment narracji

Zahava Laskier stoi przed domem ojca. Jest rok 2008, to jej jedyna wizyta
w Będzinie. Ojciec Zahavy, Jakub, przed wojną mieszkał z żoną i córką
w eleganckiej kamienicy w centrum miasta. Zahava nigdy nie poznała tych kobiet.
Stojąc przed walącą się ruderą o zabetonowanych oknach, Zahava mówi:
„Przyjeżdżając tutaj, zastanawiałam się, czy dom mojego ojca został zburzony,
czy przetrwał. Nie spodziewałam się jednak czegoś pomiędzy”.

\hangindent3cm
\hangafter=0
[nagranie głosu Zahavy, mówiącej po angielsku z silnym hebrajskim akcentem]\\
\emph{I wondered if this house is still here or was it destroyed. I didn't
suspect it can be somewhere in between.}

Zahavę oprowadza po Będzinie pan Adam. Przedstawia ją miejscowym politykom
i notablom. Zahavę witają kolejno: pan Andrzej, pan Krzysztof, pan Łukasz
i pani Stanisława. Jakub Laskier, ojciec Zahavy, przed wojną także był ważną
osobistością w mieście. Opowiadał córce o swoich współpracownikach w radzie
miasta. Zahava pamięta: pan Tewel był wysoki i~przystojny, pan Icchak lubił
podnosić głos, Abram i Mordechaj zawsze głosowali tak samo, Cyrele mieszkała
w sąsiedztwie, a Mosze sprzedawał najlepsze koszerne mięso w Zagłębiu.

\hangindent3cm
\hangafter=0
[nagranie]\\
ZAWIADAMIAMY WSZYSTKICH Ob. O NIEUCZĘSZCZANIE PO ZAKUP MIĘSNY OD TEGO PLUGAWEGO
I WSTRĘTNEGO ŻYDA PONIEWAŻ JEST STRASZNYM NIECHLUJEM I PAPRA WYROBY MIĘSNE;
A PODRU\-GIE NIE POWINNIŚMY MY POLACY PRZEZ TEN KRUTKI CZAS ICH PANOWANIA
W KTURYM JESZCZE ŻYJĄ DAWAĆ ŻADNEGO UTARGOWANIA BO RASA ŻYDOWSKA JEST BARDZO
ROZMNAŻAJACA A MY JUŻ MAMY DOSYĆ TYCH ŻYDOWSKICH IWREJÓW I POWINNIŚMY SIĘ
STARAĆ ABY ICH WKRUTKIM CZASIE WYTĘPIC.!

\hangindent5cm
\hangafter=0
WYDZIAŁ ZWIĄSKUW POTAJEMNYCH \\
PREZES ZWIĄSKU

Pobliska aleja Hugona Kołłątaja to gruba linia bezlitośnie przekreślająca
historycznie żydowskie dzielnice: Stare Miasto i Podzamcze. Bezlitośnie,
ponieważ by ją wybudować, w 1971 roku, komunistyczne władze zdecydowały
o wyburzeniu całej północnej pierzei miejskiego Rynku. Od tamtej pory Rynek
nazywa się placem Kazimierza Wielkiego. W 1358 roku Kazimierz Wielki nadał
Będzinowi prawa miejskie. Ironia była w tym wypadku niezamierzona.

\hangindent3cm
\hangafter=0
[Nika]\\
Przyjeżdżając tutaj, zastanawiałam się, czy dom mojego ojca został zburzony,
czy przetrwał. Nie spodziewałam się jednak czegoś pomiędzy.

Zahava nie rozumie, dlaczego boisko żydowskiego klubu Hakoach, a przede
wszystkim sąsiedni kirkut w latach pięćdziesiątych zostały zalane asfaltem. Nie
może zrozumieć, że kirkut nie był już wtedy nikomu potrzebny, a najpilniejszą
bieżącą potrzebą była budowa nowej zajezdni autobusowej. Pan Adam, przewodnik
Zahavy, uratował fragmenty zniszczonych macew z rampy kolejowej, którą
zbudowano z nich w~tym samym czasie, co zajezdnię. Wcześniej z tego miejsca
odjeżdżały transporty do Auschwitz. Niemcy decydowali o tym, kogo wywieźć,
a~kogo zostawić, podczas selekcji na boisku Hakoachu.

\hangindent3cm
\hangafter=0
[\textmd{nagrane}]\\
Derby Będzina. BKS kontra Hakoach -- 0:1\\
Hakoach znów górą. Piłkarze BKS-u po raz kolejny nie mieli najmniejszych szans.
Czy kiedykolwiek ktoś pokona żydowskich piłkarzy?

Nazewnictwo w Będzinie zawsze było bezlitosne. Zajezdnię autobusową na
współczesnej mapie można znaleźć przy ulicy Wyzwolenia. Hakoach po hebrajsku to
,,siła''. Pełna nazwa polskiego BKS-u to --- Będziński Klub Sportowy
,,Sarmacja''. 

\end{document}
