\documentclass[11pt,a4paper,oneside]{article}
\usepackage[polish]{babel}
\usepackage{polski}

\usepackage[utf8]{inputenc}
\usepackage{DejaVuSerifCondensed}
\usepackage[T1]{fontenc}
\usepackage{indentfirst}
\frenchspacing

\setlength{\parindent}{1em}
% \setlength{\parskip}{2.5em}
\linespread{1.15}
\setlength{\emergencystretch}{3em}  % prevent overfull lines

\pagestyle{empty}

\begin{document}
\noindent
\textbf{nil. słuchowisko}\\
szkic osi narracyjnej
\\

\noindent
Ksiądz Zawadzki czuł się skonfudowany swoją nową rolą. Fakt,
w prowadzeniu spisów parafialnych był doświadczony, wiedział jak
utrzymać porządek w tych wszystkich tabelach. Teraz to jednak coś
innego. Teraz ma prowadzić spisy powszechne --- do archiwów
parafialnych trafią innowiercy. A tych w mieście jest blisko
osiemdziesiąt procent. Żydzi, głównie Żydzi. 

Może być skonfudowany, władzy carskiej i tak się nie sprzeciwi. Poza
tym, może lepiej, żeby spisy prowadził polski, katolicki ksiądz, niż
rosyjki urzędnik z nadania lub rada kahału. To jednak polskie miasto.

Rosyjska władza zleciła katolickiemu księdzu prowadzenie urzędu
stanu cywilnego. Dzięki niemu wiemy, że tych Żydów naprawdę było
osiemdziesiąt procent. 

\vspace{5em}
\noindent
Ksiądz Zawadzki --- proboszcz w będzińskiej parafii św. Trójcy.
Z góry zamkowej, gdzie stoi jego kościół, widzi całe miasto. 
Carska administracja przydziela mu zatem zadanie. Będzie 
prowadził spisy ludności Będzina. Spisy powszechne, obejmujące
katolików, prawosławnych, Żydów. Żydów, zwłaszcza Żydów ---
już za chwilę wypełnią osiedziesiąt procent tabel. 


\end{document}
