\documentclass[11pt,a4paper,oneside]{article}
\usepackage[polish]{babel}
\usepackage{polski}

\usepackage[utf8]{inputenc}
\usepackage{DejaVuSerifCondensed}
\usepackage[T1]{fontenc}
\usepackage{indentfirst}
\frenchspacing

\usepackage{marginnote}
\usepackage[backgroundcolor=white, linecolor=white, bordercolor=white, textsize=scriptsize]{todonotes}
\usepackage[top=3cm, bottom=4cm, outer=7cm, inner=3cm, heightrounded, marginparwidth=4cm, marginparsep=1cm]{geometry}

\setlength{\parindent}{0em}
\setlength{\parskip}{2.5em}
\linespread{1.15}
\setlength{\emergencystretch}{3em}  % prevent overfull lines

\pagestyle{empty}

\begin{document}
\noindent
\textbf{nil. słuchowisko}\\
szkic osi narracyjnej

\noindent
Ksiądz Zawadzki --- proboszcz w będzińskiej parafii św. Trójcy.
Z~góry zamkowej, gdzie stoi jego kościół, widzi całe miasto.  Z tej
okazji carska administracja przydziela mu zadanie. Będzie prowadził
spisy ludności Będzina.  Spisy powszechne, obejmujące katolików,
prawosławnych, Żydów\dots{} Żydów, zwłaszcza Żydów --- już za chwilę
wypełnią mu osiemdziesiąt procent parafialnych archiwów. 

\todo{Stylistyka celowa. Chodzi o zderzenie tego akapitu
z~faktograficzną narracją następnego. Tekst mówiony ironicznie.}
Ksiądz zamyśla się czasami patrząc na stojące obok kościoła ruiny
zamku.  Wyobraża sobie Będzin w okresie świetności, gdy Kazimierz
Wielki zostawił go murowanym, a szum rzeki zakłócał tętent końskich
kopyt i łopot rycerskich sztandarów\dots{}

\hangindent3cm
\hangafter=0
\todo{Fragmenty kursywą czytane są przez kogoś innego.}
\footnotesize{\emph{Za  około sto trzydzieści lat podczas sesji rady
miasta zatęskni tak również Jan Kowalski --- radny z ramienia
Towarzystwa Przyjaciół Będzina.}}

\normalsize
Czasy się jednak zmieniły. Dziś Będzin to Jerozolima Zagłębia. Do miasta
zjeżdżają tysiące Żydów z całego Królestwa Polskiego. Inwestują
w przemysł, zakładają organizacje, kłócą się i razem modlą. Rada
kahału właśnie zebrała środki na budowę wielkiej murowanej synagogi.
Wybudują ją u samego podnóża zamku. U~podnóża jego ruin.

Od otwarcia synagogi rozpoczyna się długi okres samotności będzińskich
proboszczy. Widzą z góry miasto i wiedzą, że nie należy już do nich.
Są katolikami. Należą do mniejszości. Z tęsknotą patrzą na ruiny zamku w nadziei,
że odzyska jeszcze swój blask i dominującą pozycję w mieście. 

\hangindent3cm
\hangafter=0
\footnotesize{\emph{Tak się stanie w roku 1956, kiedy władze miasta
odbudują zamek na podstawie projektu architekta Zygmunta Gawlika. Od
tej pory o Będzinie znów będzie się mówiło, że to Królewski Gród.}}

\normalsize
Nie tylko proboszczowi sytuacja ta się nie podoba. Żydzi obsiedli też
stołki w Radzie Miasta. Są większością w mieście, więc mają większość
w radzie. Jeśli jednak stracą pozycję w jednym, stracą i~w~drugim. Trzeba
tylko wymyśleć, jak poradzić sobie z~ich większościowym statusem. 

\todo{Tu kończy sie wstęp. Pauza w narracji. Solo muzy.}
\hrule

W roku 1931 znalazł sie sprytny z pomysłem. Siedział kiedyś na Wzgórzu
Zamkowym i przyglądał się okolicznym wsiom. Carskie przepisy zabroniły
Żydom osadzania się poza miastem, więc tam dostrzegł prawdziwą ostoję
polskości. A gdyby tak włączyć je w granice miasta? 

Jedna decyzja, by z dnia na dzień Żydzi znów stali się mniejszością.
Jeden podpis, by żydowskie miasto stało się zaledwie jedną z dzielnic
miasta.  Dzielnicą żydowską, tą zamieszkałą przez Żydów. 

Teraz będą mogli mieć swoją reprezentację, będą mogli startować
w wyborach w interesie będzińskich Żydów. Będą mogli zadbać o własną
sytuację i pozycję w mieście. 

Decyzję podjęto. Tym razem się udało. 

Tym razem, bo pierwsze próby spolszczenia Będzina podjęto trzynaście 
lat wcześniej. Zaczęło się w roku 1917. Pruski okupant
rozpisał wybory samorządowe, po których Rada Miasta składała się
w całości z Żydów. Kiedy rok później Polska odzyskała niepodlełość,
ktoś wyżej niż miasto zadbał o pospolitość i polskość Będzina,
wstrzykując do Rady kilku ,,Polaków''. Żydowscy radni złożyli
mandaty, zrobiło się zamieszanie, pozostał smród. Miało być
patriotycznie, a wyszło jak zawsze.

Teraz to jednak coś innego. Włączenie okolicznych wsi jest
niepodważalne --- mamy jedną parafię, jeden targ, robimy w  tych
samych kopalniach. Kto by zauważył, że to pomysł sprytnego, gdzięki
któremu przestanie czuć się w mieście samotny. Kto by zauważył, 
że pomysł sprytnego komuś innemu odebierze poczucie \emph{bycia u siebie}.

To właśnie jest skuteczna strategia. Bez siły, bez jawnej 
przemocy. Coś się zmieniło, bo taka jest kolej rzeczy. 
Ciężko byłoby nawet powiedzieć, co. 

\end{document}
