\documentclass[11pt,a4paper,oneside]{article}
\usepackage[polish]{babel}
\usepackage{polski}

\usepackage[utf8]{inputenc}
\usepackage{DejaVuSerifCondensed}
\usepackage[T1]{fontenc}
\usepackage{indentfirst}
\frenchspacing

\usepackage{multicol}
\usepackage{marginnote}
\usepackage[backgroundcolor=white, linecolor=white, bordercolor=white, textsize=scriptsize]{todonotes}
\usepackage[top=3cm, bottom=4cm, outer=7cm, inner=3cm, heightrounded, marginparwidth=4cm, marginparsep=1cm]{geometry}

\setlength{\parindent}{0em}
\setlength{\parskip}{1.0em}
\linespread{1.15}
\setlength{\emergencystretch}{3em}  % prevent overfull lines

\pagestyle{empty}

\begin{document}
\noindent
\textbf{nil. słuchowisko}\\
fragmenty dodatkowe
 
\todo{To tylko notatka pokazująca zasadę budowania sceny. Trzeba wypełnić ją
realnymi danymi.

\vspace{1em}
Role/święta:\\ 
A -- żydowskie\\ 
B -- cerkiewne\\ 
C -- katolickie}
\line(1,0){320}

\textbf{Narrator}\\
Kalendarz na rok 1924. Wydawnictwo Cechu Drukarskiego
,,Zagłębie''. Nakład 15 tys. egz.

Styczeń.

\textbf{C}\\
Pierwszy. Nowy Rok. \\
Szósty. Trzech Króli.

\textbf{B}\\
Siódmy. Rożdiestwo Christowo. \\
Dziewiętnasty. Kreszczenije Hospodnie

\textbf{A, B, C}\\
Dwudziesty pierwszy. Dzień Babci.

\textbf{A, B, C}\\
Dwudziesty drugi. Dzień Dziadka.

\textbf{Narrator}\\
Luty.

\textbf{C}\\
Drugi. Ofiarowanie Pańskie.

\textbf{B}\\
Piętnasty. Sretienije Hospodnie.

\textbf{A}\\
Dwudziesty ósmy. Purim.

\dots{}

\line(1,0){320}

\newpage
\line(1,0){320}

\textbf{Narrator}\\
Po ponad ośmiu miesiącach okupacji Będzina, 22.05.1940 roku, komendant 45
rewiru policyjnego pisze do komendanta V odcinka policji ochronnej:


\todo{Tekst A i B mówione są symultanicznie jak przy tłumaczeniu. Tekst
A mówiony aktorsko, tekst B lektorsko.}
\textbf{A}\\
\emph{tekst po niemiecku}


\textbf{B}\\
\emph{Zwracam się z prośbą o nadesłanie do miasta Będzina większej ilości
funkcjonariuszy policji, głównie cywilnych. Żydzi się nas nie słuchają. Łamią
stawiane zakazy, jeżdżą po aryskiej stronie w tramwajach i autobusach, nie
ustępują miejsca na ulicy obywatelom Trzeciej Rzeszy, nie chcą nosić opaski
z Gwiazdą Dawida oraz nie przestrzegają godziny policyjnej. Próby zatrzymania
często kończą się na pobiciu funkcjonariuszy przez grupki żydowskie. Jesteśmy 
bezsilni. Prosimy o pomoc.}


\textbf{Narrator}\\
Niemcy chcieli być sprytni i pomysłowi. Chcieli tak organizować świat, 
żeby wszystko działało jak w zegarku. W rzeczywistości jednak zegarek
trzeba było nagręcić. Nazistom mechanizm zawsze nakręcała II Pancerna. 




\end{document}
