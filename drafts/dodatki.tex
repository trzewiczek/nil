\documentclass[11pt,a4paper,oneside]{article}
\usepackage[polish]{babel}
\usepackage{polski}

\usepackage[utf8]{inputenc}
\usepackage{DejaVuSerifCondensed}
\usepackage[T1]{fontenc}
\usepackage{indentfirst}
\frenchspacing

\usepackage{multicol}
\usepackage{marginnote}
\usepackage[backgroundcolor=white, linecolor=white, bordercolor=white, textsize=scriptsize]{todonotes}
\usepackage[top=3cm, bottom=4cm, outer=7cm, inner=3cm, heightrounded, marginparwidth=4cm, marginparsep=1cm]{geometry}

\setlength{\parindent}{0em}
\setlength{\parskip}{1.0em}
\linespread{1.15}
\setlength{\emergencystretch}{3em}  % prevent overfull lines

\pagestyle{empty}

\begin{document}
\noindent
\textbf{nil. słuchowisko}\\
fragmenty dodatkowe
 
\line(1,0){320}

\textbf{Narrator}\\
Kalendarz na rok 1924. Nadkładem Spółdzielni Drukarskiej 
,,Zagłębie''. Styczeń.

\textbf{C}\\
Pierwszy. Nowy Rok. \\
Szósty. Trzech Króli.

\textbf{B}\\
Jedenasty. Boże Narodzenie.

\textbf{A, B, C}\\
Dwudziesty szósty. Dzień Babci.

\textbf{A, B, C}\\
Dwudziesty siódmy. Dzień Dziadka.

\textbf{Narrator}\\
Luty.

\textbf{A}\\
Dwudziesty ósmy. Purim.

\line(1,0){320}


\end{document}
