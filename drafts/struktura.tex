\documentclass[10pt,a4paper,oneside]{article}
\usepackage[polish]{babel}
\usepackage{polski}

\usepackage[utf8]{inputenc}
\usepackage{DejaVuSerifCondensed}
\usepackage[T1]{fontenc}
\usepackage{indentfirst}
\frenchspacing

\usepackage{color}
\usepackage[usenames,dvipsnames,svgnames,table]{xcolor}
\definecolor{extras}{HTML}{888888}
\definecolor{sound}{HTML}{A05A2C}
\definecolor{loss}{HTML}{216778}
\definecolor{grey}{HTML}{cccccc}

\usepackage{marginnote}
\usepackage[backgroundcolor=white, linecolor=white, bordercolor=white, textsize=footnotesize]{todonotes}
\usepackage[top=3cm, bottom=4cm, outer=7cm, inner=3cm, heightrounded, marginparwidth=4cm, marginparsep=1cm]{geometry}

\setlength{\parindent}{0em}
\setlength{\parskip}{2.5em}
\linespread{1.15}
\setlength{\emergencystretch}{3em}  % prevent overfull lines

\pagestyle{empty}

\begin{document}
\noindent
\textbf{nil. słuchowisko}\\
struktura osi narracyjnej


% WPROWADZENIE
\todo{Proboszcz z parafii św. Trójcy widzi, tęskni, ma nadzieję.}
{\large Wprowadzenie}\\
Będzin z końca XIX wieku. Miasto w większości Żydowskie.
Jerozolima Zagłębia.  Późnośredniowieczne korzenie miasta.
Zaznaczona rycerska linia w tożsamości Będzina. 


% MUZYKA Z ZAHAVĄ
\todo{Nagrany przez Nikę fragment z Zahavą pojawia
się w muzie bez odjaśnienia.}
\hangindent2cm
\hangafter=0
{\color{sound}
{\large \emph{Muzyka}}
}


% WYBORY 1917
\todo{Sprytny z Warszawy.}
{\large Wybory 1917}\\
Próba manipulacji wyborami do samorządów na szczeblu
centralnym. Podział okręgów wyborczych na grupy zawodowe,
a nie terytorialnie. W Będzinie kontrefektywne.  Bojkot
polskich partii. Żydzi stanowią sto procent Rady.


% IMIONA RADNYCH I
\hangindent2cm
\hangafter=0
{\large\color{loss} Imiona radnych I}


% WYBORY 1931
\todo{Sprytny z Będzina.}
{\large Wybory 1931}\\
Kolejna próba manipulacji wyborami. Włączenie okolicznych
wsi. Prawo carskie a zachowanie polskości wsi. Zmiana
statystyk. ,,Powstanie'' dzielnicy żydwoskiej.

\hangindent2cm
\hangafter=0
{\large\color{loss} Imiona radnych II}

{\color{extras}
{\large Prawo dotyczące karczm}\\
Zakaz sprzedaży alkoholu przez Żydów przy zakładach pracy.
Ochrona polskich robotników przez rozpiciem. Zakaz pracy
Żydów w tych zakładach. Możliwość posiadania tych samych
zakładów. 
}

{\color{extras}
{\large Nike na pl. 3 Maja}\\
Pomnik 11 Pułku Piechoty ustawiony na głównym placu miasta.
Centralny punkt chasydzkiej dzialnicy Będzina.  Nike gorszy
Żydów. Wysadzony przez nazistów w 1939.  Żarty na ten temat.
Pomnik w innym kształcie odbudowany w III RP. Towarzystwo
Przyjaciół Będzina chce Nike --- symbolu miasta.
}

\hangindent2cm
\hangafter=0
{\large\color{loss} List komendanta policji.}

{\large Godzina policyjna}\\
Niemcy wprowadzają godzinę policyjną. Zaczyna się o 19.00.
Latem ulice centrum pustoszeją, choć jest zupełnie widno.
,,Żydowska dzielnica'' pozbawiona Żydów. 

\todo{Tutaj muzyka rozkręca się do bardzo dynamicznej. Na
pauzie Nika przyczy ,,Godzina policyjna''. Wejście
piosenki.}
\hangindent2cm
\hangafter=0
{\color{sound}
{\large \emph{Muzyka}}
}

\hangindent2cm
\hangafter=0
{\color{sound}
{\large \emph{Piosenka o ulicach}}
}

\todo{Wybrzmienie piosenki jako łącznik}
\hangindent2cm
\hangafter=0
{\color{sound}
{\large \emph{Muzyka}}
}

\todo{Fragment opowiadany na muzie z dużym pokładem
emocjonalnym}
{\color{extras}
{\large Selekcja na boisku}\\
Getto pracuje. Selekcja niesprawnych do pracy na boisku
Hakoachu. Deszcz, panika, czarna farba z siwych włosów.
}

\hangindent2cm
\hangafter=0
{\color{sound}
{\large \emph{Piosenka o świadectwach}}
}

\todo{Dynamiczne cięcie za pomocą faktu, konkretu.}
\todo{Proboszcz na obchodach ugrywa swoje interesy.}
{\large 600-lecie Będzina}\\
Obchody 600-lecia miasta w 1958. Założyciel Będzina Hinko
Ethiopus. Mówienie o Będzinie 15 lat po pogromie Żydów.
Miasto widmo. Pałac Mieroszewskich otrzymuje nowy piec.
Dom Wójtowski otrzymuje tablicę ,,Na wieczną pamiątkę''. 
Dziś to ruina. Obchody 650-lecia w 2008.

\hangindent2cm
\hangafter=0
{\large\color{loss} Imiona radnych III}

{\large Polityka historyczna I}\\
Pamiętnik Rutki Laskier. Akademie szkolne. Inscenizacja
likwidacji getta. Smutny los naszych sąsiadów.


% LOVE SONG
\hangindent2cm
\hangafter=0
{\color{sound}
{\large \emph{Love song}}
}


% POLITYKA HISTORYCZNA II
{\large Polityka historyczna II}\\
Śmierć Rutki. Hipotetyczne pamiętniki. 


% ZAHAVA
\hangindent2cm
\hangafter=0
{\large\color{loss} Zahava}


% MIASTECZKO BEŁZ
\todo{Nika zaczyna śpiwać na wersji jidisz. Włącza się
gitara i perkusja. Elektronika buduje ścianę noisu aż do
zagłuszenia.}
\hangindent2cm
\hangafter=0
{\color{sound}
{\large \emph{Miasteczko Bełz}}
}

\end{document}
