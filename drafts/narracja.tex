\documentclass[11pt,a4paper,oneside]{article}
\usepackage[polish]{babel}
\usepackage{polski}

\usepackage[utf8]{inputenc}
\usepackage{DejaVuSerifCondensed}
\usepackage[T1]{fontenc}
\usepackage{indentfirst}
\frenchspacing

\usepackage{marginnote}
% \usepackage[backgroundcolor=white, linecolor=white, bordercolor=white, textsize=scriptsize]{todonotes}
\usepackage[top=3cm, bottom=4cm, outer=5cm, inner=5cm, heightrounded]{geometry}

\usepackage{xcolor}
\definecolor{light-gray}{gray}{0.4}
\setlength{\parindent}{0em}
\setlength{\parskip}{2.0em}
\linespread{1.15}
\setlength{\emergencystretch}{3em}  % prevent overfull lines

\pagestyle{empty}

\begin{document}
\noindent
\textbf{nil. słuchowisko}\\

\textbf{Narrator}\\
Ksiądz Adamski --- proboszcz w będzińskiej parafii św. Trójcy.
Z~góry zamkowej, gdzie stoi jego kościół, widzi całe miasto.  Z~tej
okazji carska administracja przydziela mu zadanie. Będzie prowadził
spisy ludności Będzina.  Spisy powszechne, obejmujące katolików,
prawosławnych, Żydów\dots{} 

\hangindent3cm
\hangafter=0
{\color{light-gray} \emph{Khrrra. Wchodzi pierwszy noise}}

Żydów, zwłaszcza Żydów --- już za chwilę wypełnią mu osiemdziesiąt procent
parafialnych archiwów. 

\hangindent3cm
\hangafter=0
{\color{light-gray} \emph{Muzyka śliczna, z przestrzenią. Punktowy noise pod spodem}}

{\color{light-gray} \emph{Z zadumą i tęsknotą}} \\
Ksiądz zamyśla się czasami patrząc na stojące obok kościoła ruiny
zamku.  Wyobraża sobie Będzin w okresie świetności, gdy Kazimierz
Wielki zostawił go murowanym, a szum rzeki zakłócał tętent końskich
kopyt i łopot rycerskich sztandarów\dots{}

\hangindent3cm
\hangafter=0
{\color{light-gray} \emph{Pauza w muzyce}}

\textbf{A}\\
{\color{light-gray} \emph{Reportersko}}\\
Za około sto trzydzieści lat podczas sesji rady miasta zatęskni tak również Adam
Szydłowski --- radny z ramienia Towarzystwa Przyjaciół Będzina.

\hangindent3cm
\hangafter=0
{\color{light-gray} \emph{Wraca muzyka. Dźwiękowe ślady miasta}}

\textbf{Narrator}\\
Czasy się jednak zmieniły. Dziś Będzin to Jerozolima Zagłębia. 
Do miasta zjeżdżają tysiące Żydów z całego Królestwa Polskiego.
Inwestują w przemysł, zakładają organizacje, kłócą się i razem modlą.
Rada kahału właśnie zebrała środki na budowę wielkiej murowanej
synagogi. Wybudują ją u samego podnóża zamku. U~podnóża jego ruin.

\hangindent3cm
\hangafter=0
{\color{light-gray} \emph{Muzyka przeciąga pauzę. W muzykę wplecione
nagranie:}}\\
Przyjeżdżając tutaj, zastanawiałam się, czy dom mojego ojca został
zburzony, czy przetrwał. Nie spodziewałam się jednak czegoś pomiędzy.

Od otwarcia synagogi rozpoczyna się długi okres samotności będzińskich
proboszczy. Widzą z góry miasto i wiedzą, że nie należy już do nich.
Są katolikami. Należą do mniejszości. Z tęsknotą patrzą na ruiny zamku w nadziei,
że odzyska jeszcze swój blask i dominującą pozycję w mieście. 

\hangindent3cm
\hangafter=0
{\color{light-gray} \emph{Pauza w muzyce}}

\textbf{A}\\
{\color{light-gray} \emph{Reportersko}}\\
Tak się stanie w roku 1956, kiedy władze miasta odbudują zamek na podstawie
projektu architekta Zygmunta Gawlika. Od tej pory o Będzinie znów mówi się
Królewski Gród.

\textbf{Narrator}\\
{\color{light-gray} \emph{Zmiana tonu. Czyta jak obwieszczenie}}\\
Kalendarz na rok 1924. Wydawnictwo Cechu Drukarskiego
,,Zagłębie''. Nakład 15 tys. egz. Styczeń.

\textbf{C}\\
Pierwszy. Nowy Rok. \\
Szósty. Trzech Króli.

\textbf{B}\\
Siódmy. Rożdiestwo Christowo. \\
Dziewiętnasty. Kreszczenije Hospodnie

\textbf{Narrator}\\
Luty.

\textbf{C}\\
Drugi. Ofiarowanie Pańskie.

\textbf{B}\\
Piętnasty. Sretienije Hospodnie.

\textbf{A}\\
Dwudziesty ósmy. Purim.

{\color{red} \emph{Uzupełnić pozostałe miesiące}}

\textbf{Narrator}\\
Grudzień.

\textbf{C}\\
Dwudziesty piąty. Boże Narodzenie.

\hangindent3cm
\hangafter=0
{\color{light-gray} \emph{Koniec wstępu. Pauza w narracji. Solo
muzyki. Muzyka w wybrzmieniu zawiesza się zapowiadając, że teraz
zaczyna się główna część słuchowiska.}}

\textbf{Narrator}\\
Pszczyna, Warszawa. 1916. Sprytny Numer Jeden. 

\hangindent3cm
\hangafter=0
{\color{light-gray} \emph{Wstaje} B, \emph{kłania się, perkusja odbija
ta-pum jak w telewizyjnym talkshow}}

Jest tak: 

\hangindent3cm
\hangafter=0
{\color{light-gray} \emph{Wchodzi ride --- muzyka z kryminału}}

wojna trwa już drugi rok, Prusy i Austro-Węgry wyrżnęły większość
swoich żołnierzy i cierpią na dramatyczny brak rekruta. Cierpią na
tyle, że oferują Polakom własne państwo --- byle tylko ci tłumnie ruszyli na
Rosję, byle tylko zasilili ich armie. Polacy pomysł kupili, szybko 
zorganizowali szczupłą państwowość, a rekruci... no cóż, nie tym razem.

Kiedy miesiąc później w Warszawie podejmowano decyzję o zorganizowaniu wyborów
samorządowych, Sprytny Numer Jeden siedział wysoko w ławie Tymczasowej Rady Stanu
i ubolewał nad losem polskiej prowincji. ,,Co będzie, jeśli Żydzi przejmą
rady gminne?'' --- rozpaczał. ,,Czy możemy pozwolić sobie na takie ryzyko
po stu latach niewoli?''. Pewnie nie.

Sprytny Numer Jeden zadręczał się tymi pytaniami przez wiele dni, aż
wymyślił. Strategia, którą zaproponował miała być bezbłędna. Wydawało 
mu się bowiem, że jeśli podzielić gminy na grupy zawodowe, a nie obszary,
Żydzi nie zdobędą większości w radach gminnych. 

\hangindent3cm
\hangafter=0
{\color{light-gray} \emph{Gramofonowe zerwanie muzyki}}

\hangindent3cm
\hangafter=0
{\color{light-gray} \emph{Na ciszy nagranie z offu}}\\
\emph{Obywatele! Od 1917 roku mamy istotny samorząd miejski oparty na
wyborach. Od naszej własnej dojrzałości zależy abyśmy wybrali Radę
Miejską, możliwie najlepszą, a przedewszystkiem \textbf{naszą,
polską}, jak polskiem jest miasto nasze i ukochany Kraj nasz ojczysty.
Kto nie spełni swojego obowiązku ten nie jest godzien nazwy polaka
i obywatela naszego miasta!}

\hangindent3cm
\hangafter=0
{\color{light-gray} \emph{Gramofonowe przywrócenie muzyki}}

W Będzinie nowa ordynacja wyborcza zagwarantowała Żydom siedemdziesiąt
procent miejsc w radzie jeszcze przed głosowaniem. Żydowskie partie
zaproponowały zatem polskim ugrupowaniom sojusze za gwarancję tych
siedemdziesięciu procent. Wybuchł skandal, polskie partie ogłosiły
bojkot wyborów, nie wystawiły żadnego kandydata, a Radę Gminy uformowali:

\hangindent3cm
\hangafter=0
{\color{light-gray} \emph{Pauza w muzyce}}

\textbf{A}\\
Abraham\\
Chil\\
Dawid\\
Gerszlik\\
Herman\\
Hirsch\\
Isaak\\
Izrael\\
Jacob\\
Joachim\\
Juda\\
Józef\\
Michel\\
Mojżesz\\
Moszek\\
Salomon\\
Szalum

\hangindent3cm
\hangafter=0
{\color{light-gray} \emph{Pauza}}

\textbf{Narrator}\\
Będzin. 1931. Sprytny Numer Dwa. 

\hangindent3cm
\hangafter=0
{\color{light-gray} \emph{Wstaje} C, \emph{kłania się, perkusja odbija
ta-pum}}

\hangindent3cm
\hangafter=0
{\color{light-gray} \emph{Muzyka ilustracyjna, szerokim dźwiękiem}}

Sprytny Numer Dwa siedział kiedyś na Wzgórzu Zamkowym i~przyglądał się
okolicznym wsiom. Carskie przepisy zabroniły Żydom osadzania się poza
miastem, więc to tam właśnie dostrzegł prawdziwą ostoję polskości. Tam
dostrzegł braci katolików --- we wsi szczęśliwej. Zatęsknił. Zapłakał.

\hangindent3cm
\hangafter=0
{\color{light-gray} \emph{Muzyka się urywa}}

A gdyby włączyć te wsie w granice miasta?

\hangindent3cm
\hangafter=0
{\color{light-gray} \emph{Muzyka jak z kryminału --- zaczyna sama
perkusja, powoli się nakręca}}

{\color{light-gray} \emph{Po pauzie}}\\
Jedna decyzja, by z dnia na dzień Żydzi znów stali się mniejszością.
Jeden podpis, by żydowskie miasto stało się zaledwie jedną z dzielnic
miasta.  Dzielnicą żydowską, tą --- zamieszkałą przez Żydów. 

Jedna decyzja, by mogli mieć swoją reprezentację. Jeden podpis, by
mogli startować w wyborach w imieniu będzińskich Żydów, by mogli
zadbać o własną sytuację i pozycję w mieście. 

Decyzję podjęto. Sarmacja Będzin kontra ŻKS Hakoach --- 1:0. Po
najbliższych wyborach Radę Miasta formują:

\hangindent3cm
\hangafter=0
{\color{light-gray} \emph{Pauza w muzyce}}

\textbf{A}\\
Abraham\\
Dawid

\textbf{B}\\
Franciszek

\textbf{A}\\
Gerszlik\\
Hirsch\\
Isaak

\textbf{B}\\
Jan

\textbf{A}\\
Joachim

\textbf{B}\\
Kazimierz

\textbf{A}\\
Mojżesz\\
Moszek

\textbf{B}\\
Paweł\\
Piotr

\textbf{A}\\
Salomon

\textbf{B}\\
Sławomir\\
Stanisław

\textbf{A}\\
Szalum

\textbf{B}\\
Tomasz


\hangindent3cm
\hangafter=0
{\color{light-gray} \emph{Pauza}}


\hangindent3cm
\hangafter=0
{\color{light-gray} \emph{Muzyka śliczna. Punktowy noise pod spodem}}

\textbf{Narrator}\\
Raduje się dusza księdza Zawadzkiego na myśl, że znów zamieszka
,,między swemi''. Parafialne księgi zapełnią się katolikami, 
bracia starsi staną się mniejszością, a ład powróci do miasta.

\hangindent3cm
\hangafter=0
{\color{light-gray} \emph{Pauza w muzyce}}

\textbf{A}\\
{\color{light-gray} \emph{Reportersko}}\\
Ksiądz Zawadzki nie wie, że pośmiertnie odznaczony zostanie tytułem
,,Sprawiedliwego wśród narodów świata'' za ratowanie Żydów
z podpalonej przez nazistów synagogi.

\hangindent3cm
\hangafter=0
{\color{light-gray} \emph{Plama dźwiękowa niezauważalnie narastająca
do niewysokiego poziomu. Jej obecność staje się zauważalna po nagłym
cięciu}}

\textbf{Narrator}\\
Przyłączenie okolicznych wsi do Będzina jest niepodważalne --- mamy
jedną parafię, jeden targ, robimy w  tych samych kopalniach. Kto by
zauważył, że to pomysł Sprytnego, a nie dziejowa konieczność
rozwijającego się miasta.  Kto by zauważył, że pomysł Sprytnego komuś
w mieście odebierze poczucie \emph{bycia u siebie}.  

To właśnie jest skuteczna strategia. Bez siły, bez jawnej przemocy.
Coś się zmieniło, bo taka jest kolej rzeczy.  Ciężko byłoby nawet
powiedzieć, co. Startegia jest skuteczna o tyle, o ile jej nie widać. 

\hangindent3cm
\hangafter=0
{\color{light-gray} \emph{Cięcie w muzyce}}


\textbf{Narrator}\\
Będzin. 1921--1933. Oczywisty. 

\hangindent3cm
\hangafter=0
{\color{light-gray} \emph{Wstaje} C, \emph{kłania się, perkusja odbija
ta-pum}}

{\color{red} kapelan wojskowy --- kolega probszcza --- pomnik Nike --- duma
i zwycięztwo kontra nagość kobiety --- zgorszenie Chasydów --- naziści
wysadzają pomnik}


{\color{red}
GODZINA POLICYJNA
}

\hangindent3cm
\hangafter=0
{\color{light-gray} \emph{Pauza w muzyce}}

\textbf{Narrator}\\
Po ponad ośmiu miesiącach okupacji Będzina, 22.05.1940 roku, komendant 45
rewiru policyjnego pisze do komendanta V odcinka policji ochronnej:

\textbf{A}\\
{\color{light-gray} \emph{Zrozpaczony}}\\
{\color{red} tłumaczenie na niemiecki}


\textbf{B}\\
{\color{light-gray} \emph{Symultanicznie z tekstem niemieckim, jak tłumacz}}\\
Zwracam się z prośbą o nadesłanie do miasta Będzina większej ilości
funkcjonariuszy policji, głównie cywilnych. Żydzi się nas nie słuchają. Łamią
stawiane zakazy, jeżdżą po aryskiej stronie w tramwajach i autobusach, nie
ustępują miejsca na ulicy obywatelom Trzeciej Rzeszy, nie chcą nosić opaski
z Gwiazdą Dawida oraz nie przestrzegają godziny policyjnej. Próby zatrzymania
często kończą się na pobiciu funkcjonariuszy przez grupki żydowskie. Jesteśmy 
bezsilni. Prosimy o pomoc.


\textbf{Narrator}\\
Niemcy chcieli być sprytni i pomysłowi. Chcieli tak organizować świat, 
żeby wszystko działało jak w zegarku. W rzeczywistości jednak niemiecki 
zegarek trzeba było nagręcić. Nazistom mechanizm zawsze nakręcała
II Pancerna. 

\hangindent3cm
\hangafter=0
{\color{light-gray} \emph{Agresywan abstrakcyjna muzyka, dużo noisu}}

\hangindent3cm
\hangafter=0
{\color{light-gray} \emph{Nagłe cięcie w muzyce}}

\textbf{A}\\
{\color{light-gray} \emph{Krzyczy}}\\
GODZINA POLICYJNA!

\hangindent3cm
\hangafter=0
{\color{light-gray} \emph{Na raz muzyka wraca do poprzedniego poziomu
dynamicznego, lecz zorganizowana}}

{\color{red}
PIOSENKA O ULICACH

NIEWYSŁOWIONE ŚWIADECTWO AUSCHWITZ

600-LECIE MIASTA

POLITYKA HISTORYCZNA BĘDZINA
}

\hangindent3cm
\hangafter=0
{\color{light-gray} \emph{Z offu ,,Miasteczko Bełz'' w wykonaniu Adama Astona. Od refrenu nagranie zaczyna schodzić na dalszy plan (aż w końcu znika). Od refrenu włącza się gitara i wokal na scenie. Od końca refrenu narasta powoli ściana noisu aż do zupełnego zakrycia gitary i głosu. Na koniec piosenki noise się ucina.}}

\end{document}
