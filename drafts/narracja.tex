\documentclass[11pt,a4paper,oneside]{article}
\usepackage[polish]{babel}
\usepackage{polski}

\usepackage[utf8]{inputenc}
\usepackage{DejaVuSerifCondensed}
\usepackage[T1]{fontenc}
\usepackage{indentfirst}
\frenchspacing

\usepackage{marginnote}
% \usepackage[backgroundcolor=white, linecolor=white, bordercolor=white, textsize=scriptsize]{todonotes}
\usepackage[top=3cm, bottom=4cm, outer=5cm, inner=5cm, heightrounded]{geometry}

\usepackage{xcolor}
\definecolor{light-gray}{gray}{0.4}
\setlength{\parindent}{0em}
\setlength{\parskip}{2.0em}
\linespread{1.15}
\setlength{\emergencystretch}{3em}  % prevent overfull lines

\pagestyle{empty}

\begin{document}
\noindent
\textbf{nil. słuchowisko}\\
szkic osi narracyjnej

\textbf{Narrator}\\
Ksiądz Adamski --- proboszcz w będzińskiej parafii św. Trójcy.
Z~góry zamkowej, gdzie stoi jego kościół, widzi całe miasto.  Z~tej
okazji carska administracja przydziela mu zadanie. Będzie prowadził
spisy ludności Będzina.  Spisy powszechne, obejmujące katolików,
prawosławnych, Żydów\dots{} 

\hangindent3cm
\hangafter=0
{\color{light-gray} \emph{Khrrra. Wchodzi pierwszy noise}}

Żydów, zwłaszcza Żydów --- już za chwilę wypełnią mu osiemdziesiąt procent
parafialnych archiwów. 

\hangindent3cm
\hangafter=0
{\color{light-gray} \emph{Muzyka ślicznie i z przestrzenią}}

{\color{light-gray} \emph{Z zadumą i tęsknotą}} \\
Ksiądz zamyśla się czasami patrząc na stojące obok kościoła ruiny
zamku.  Wyobraża sobie Będzin w okresie świetności, gdy Kazimierz
Wielki zostawił go murowanym, a szum rzeki zakłócał tętent końskich
kopyt i łopot rycerskich sztandarów\dots{}

\textbf{A}\\
{\color{light-gray} \emph{Reportersko}}\\
Za około sto trzydzieści lat podczas sesji rady miasta zatęskni tak również Adam
Szydłowski --- radny z ramienia Towarzystwa Przyjaciół Będzina.

\textbf{Narrator}\\
Czasy się jednak zmieniły. Dziś Będzin to Jerozolima Zagłębia. 

\hangindent3cm
\hangafter=0
{\color{light-gray} \emph{W muzyce dźwiękowe ślady miasta}}

Do miasta zjeżdżają tysiące Żydów z całego Królestwa Polskiego.
Inwestują w przemysł, zakładają organizacje, kłócą się i razem modlą.
Rada kahału właśnie zebrała środki na budowę wielkiej murowanej
synagogi. Wybudują ją u samego podnóża zamku. U~podnóża jego ruin.

\hangindent3cm
\hangafter=0
{\color{light-gray} \emph{Muzyka przeciąga pauzę}}

Od otwarcia synagogi rozpoczyna się długi okres samotności będzińskich
proboszczy. Widzą z góry miasto i wiedzą, że nie należy już do nich.
Są katolikami. Należą do mniejszości. Z tęsknotą patrzą na ruiny zamku w nadziei,
że odzyska jeszcze swój blask i dominującą pozycję w mieście. 

\textbf{A}\\
{\color{light-gray} \emph{Reportersko}}\\
Tak się stanie w roku 1956, kiedy władze miasta odbudują zamek na podstawie
projektu architekta Zygmunta Gawlika. Od tej pory o Będzinie znów będzie się
mówiło, że to Królewski Gród.

\hangindent3cm
\hangafter=0
{\color{light-gray} \emph{Pauza w muzyce}}

\textbf{Narrator}\\
{\color{light-gray} \emph{Zmiana tonu. Czyta jak obwieszczenie}}\\
Kalendarz na rok 1924. Wydawnictwo Cechu Drukarskiego
,,Zagłębie''. Nakład 15 tys. egz. Styczeń.

\textbf{C}\\
Pierwszy. Nowy Rok. \\
Szósty. Trzech Króli.

\textbf{B}\\
Siódmy. Rożdiestwo Christowo. \\
Dziewiętnasty. Kreszczenije Hospodnie

\textbf{Narrator}\\
Luty.

\textbf{C}\\
Drugi. Ofiarowanie Pańskie.

\textbf{B}\\
Piętnasty. Sretienije Hospodnie.

\textbf{A}\\
Dwudziesty ósmy. Purim.

\dots{}

\textbf{Narrator}\\
Grudzień.

\textbf{C}\\
Dwudziesty piąty. Boże Narodzenie.

\hangindent3cm
\hangafter=0
{\color{light-gray} \emph{Koniec wstępu. Pauza w narracji. Solo muzyki}}

\textbf{Narrator}\\
Pszczyna, Warszawa. 1916. Sprytny Numer Jeden. 

\hangindent3cm
\hangafter=0
{\color{light-gray} \emph{Wstaje} B, \emph{kłania się, perkusja odbija
ta-pum jak w telewizyjnym talkshow}}

Jest tak: 

\hangindent3cm
\hangafter=0
{\color{light-gray} \emph{Wchodzi ride --- muzyka z kryminału}}

wojna trwa już drugi rok, Prusy i Austro-Węgry wyrżnęły większość
swoich żołnierzy i cierpią na dramatyczny brak rekruta. Cierpią na
tyle, że oferują Polakom własne państwo --- byle tylko ci tłumnie ruszyli na
Rosję, byle tylko zasilili ich armie. Polacy pomysł kupili, szybko 
zorganizowali szczupłą państwowość, a rekruci... no cóż, nie tym razem.

Kiedy miesiąc później w Warszawie podejmowano decyzję o zorganizowaniu wyborów
samorządowych, Sprytny Numer Jeden siadział w ławie Tymczasowej Rady Stanu
i ubolewał nad losem polskiej prowincji. ,,Co będzie, jeśli Żydzi przejmą
rady gminne?'' --- rozpaczał. ,,Czy możemy pozwolić sobie na takie ryzyko,
po stu latach niewoli?''. Pewnie nie.

Sprytny Numer Jeden zadręczał się tymi pytaniami przez wiele dni, aż
wymyślił. Strategia, którą zaproponował miała być zbawienna. Wydawało 
mu się bowiem, że jeśli podzielić gminy na grupy zawodowe, a nie obszary,
Żydzi nie zdobędą większości w radach gminnych. Cóż, nie tym razem!

\hangindent3cm
\hangafter=0
{\color{red}Z offu tekst odezwy z Częstochowy}

W Będzinie nowa ordynacja wyborcza zagwarantowała Żydom siedemdziesiąt
procent miejsc w radzie jeszcze przed głosowaniem. Żydowskie partie
zaproponowały zatem polskim ugrupowaniom sojusze za gwarancję tych
siedemdziesięciu procent. Wybuchł skandal, polskie partie ogłosiły
bojkot wyborów, nie wystawiły żadnego kandydata, a Radę Gminy uformowali:

\hangindent3cm
\hangafter=0
{\color{light-gray} \emph{Pauza w muzyce}}


\hangindent3cm
\hangafter=0
{\color{red}Imiona żydowskich radnych}


\textbf{Narrator}\\
Będzin. 1931. Sprytny Numer Dwa. 

\hangindent3cm
\hangafter=0
{\color{light-gray} \emph{Wstaje} C, \emph{kłania się, perkusja odbija
ta-pum jak w telewizyjnym talkshow}}

\hangindent3cm
\hangafter=0
{\color{light-gray} \emph{Muzyka ilustracyjna, szerokim dźwiękiem}}

Sprytny Numer Dwa siedział kiedyś na Wzgórzu Zamkowym i przyglądał się
okolicznym wsiom. Carskie przepisy zabroniły Żydom osadzania się poza
miastem, więc tam dostrzegł prawdziwą ostoję polskości. Tam dostrzegł
braci katolików. A gdyby włączyć te wsie w granice miasta?  

\hangindent3cm
\hangafter=0
{\color{light-gray} \emph{Na pytanie muzyka się urywa}}

\hangindent3cm
\hangafter=0
{\color{light-gray} \emph{Muzyka jak z kryminału --- powolo nakręca się}}

{\color{light-gray} \emph{Po pauzie}}\\
Jedna decyzja, by z dnia na dzień Żydzi znów stali się mniejszością.
Jeden podpis, by żydowskie miasto stało się zaledwie jedną z dzielnic
miasta.  Dzielnicą żydowską, tą zamieszkałą przez Żydów. 

Teraz będą mogli mieć swoją reprezentację, będą mogli startować
w wyborach w interesie będzińskich Żydów. Będą mogli zadbać o własną
sytuację i pozycję w mieście. 

Decyzję podjęto. Sarmacja Będzin kontra ŻKS Hakoach --- 1:0. Radę
miasta formują:

\hangindent3cm
\hangafter=0
{\color{red}Imiona polskich i żydowskich radnych}


\vspace{2cm}

KSIĄDZ ZAWADZKI CIESZY SIĘ

Teraz to jednak coś innego. Włączenie okolicznych wsi jest
niepodważalne --- mamy jedną parafię, jeden targ, robimy w  tych
samych kopalniach. Kto by zauważył, że to pomysł sprytnego, gdzięki
któremu przestanie czuć się w mieście samotny. Kto by zauważył, 
że pomysł sprytnego komuś innemu odebierze poczucie \emph{bycia u siebie}.

To właśnie jest skuteczna strategia. Bez siły, bez jawnej 
przemocy. Coś się zmieniło, bo taka jest kolej rzeczy. 
Ciężko byłoby nawet powiedzieć, co. 

\end{document}
