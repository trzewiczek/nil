\documentclass[11pt,a4paper,oneside]{article}
\usepackage[polish]{babel}
\usepackage{polski}

\usepackage[utf8]{inputenc}
\usepackage{DejaVuSerifCondensed}
\usepackage[T1]{fontenc}
\usepackage{indentfirst}
\frenchspacing

\widowpenalty=10000
\clubpenalty=10000

\usepackage{marginnote}
\usepackage[backgroundcolor=white, linecolor=white, bordercolor=white, textsize=footnotesize]{todonotes}
\usepackage[top=3cm, bottom=4cm, outer=4.5cm, inner=4.5cm, heightrounded, marginparwidth=0cm, marginparsep=0cm]{geometry}

\usepackage{xcolor}
\definecolor{light-gray}{gray}{0.4}
\definecolor{konrad}{rgb}{0.16, 0.27, 0.46}
\definecolor{krzysiek}{rgb}{0.27, 0.46, 0.16}
\setlength{\parindent}{0em}
\setlength{\parskip}{2.0em}
\linespread{1.15}
\setlength{\emergencystretch}{3em}  % prevent overfull lines

\pagestyle{empty}

\begin{document}
\noindent
\textbf{nil. słuchowisko}

Ksiądz Adamski --- proboszcz w będzińskiej parafii św. Trójcy.  Z~góry
zamkowej, gdzie stoi jego kościół, widzi całe miasto.  Z~tej okazji
carska administracja przydziela mu zadanie. Będzie prowadził spisy
ludności Będzina.  Spisy powszechne, obejmujące katolików,
prawosławnych, Żydów\dots{} zwłaszcza Żydów --- już za chwilę wypełnią
mu przecież osiemdziesiąt procent parafialnych archiwów. 

Ksiądz zamyśla się czasami patrząc na stojące obok kościoła ruiny
średniowiecznego zamku. Wyobraża sobie Będzin w okresie świetności ---
wtedy gdy Kazimierz Wielki zostawił go murowanym. Szum rzeki zakłóca
tętent końskich kopyt i łopot rycerskich sztandarów\dots{}

\hangindent2cm
\hangafter=0
Za około sto trzydzieści lat podczas sesji rady miasta zatęskni tak również Adam
Szydłowski --- radny z~ramienia Towarzystwa Przyjaciół Będzina.

Czasy się jednak zmieniły. Dziś Będzin to Jerozolima Zagłębia.  Do
miasta zjeżdżają tysiące Żydów z całego Królestwa Polskiego.
Inwestują w przemysł, zakładają organizacje i partie polityczne, kłócą
się i razem modlą.  Rada kahału właśnie zebrała środki na budowę
wielkiej murowanej synagogi. Wybudują ją u samego podnóża zamku.
U~podnóża jego ruin.

\hangindent2cm
\hangafter=0
\emph{Przyjeżdżając tutaj, zastanawiałam się, czy dom mojego ojca został
zburzony, czy przetrwał. Nie spodziewałam się jednak czegoś pomiędzy.}

Od otwarcia synagogi rozpoczyna się długi okres samotności będzińskich
proboszczy. Widzą z góry miasto i wiedzą, że nie należy już do nich.
Są katolikami. Należą do mniejszości. Z tęsknotą patrzą na ruiny zamku
w nadziei, że odzyska jeszcze swój blask i dominującą pozycję
w mieście. 

\hangindent2cm
\hangafter=0
Tak się stanie w roku 1956, kiedy władze miasta odbudują zamek na
podstawie projektu architekta Zygmunta Gawlika. Od tej pory o Będzinie
znów mówi się, że to Królewski Gród.

Kalendarz na rok 1924. Wydawnictwo Cechu Drukarskiego ,,Zagłębie''.
Nakład 15 tys. egz. 

Styczeń.

% --> STYCZEŃ
\textbf{B}\\
Pierwszy. Świętej Bożej Rodzicielki. \\
Szósty. Trzech Króli.

\textbf{C}\\
Dwudziesty szósty. Tu bi-Szwat. % koniec zimy, sadzenie drzew

% --> LUTY
\textbf{Narrator}\\
Luty.

\textbf{B}\\
Drugi. Ofiarowanie Pańskie.

\textbf{C}\\
Dwudziesty czwarty. Purim.

% --> MARZEC
\textbf{Narrator}\\
Marzec.

\textbf{C}\\
Dwudzesty szósty. Pesach.

% --> KWIECIEŃ
\textbf{Narrator}\\
Kwiecień.

\textbf{C}\\
Dwudzesty szósty. Lag baOmer.

% --> MAJ
\textbf{Narrator}\\
Maj.

\textbf{B}\\
Dwunasty. Wniebowstąpienie.

\textbf{C}\\
Piętnasty. Szawuot.

\textbf{B}\\
Trzydziesty. Boże Ciało.

% --> SIERPIEŃ
\textbf{Narrator}\\
Sierpień.

\textbf{B}\\
Piętnasty. Wniebowzięcie Najświętszej Maryi Panny.

% --> WRZESIEŃ
\textbf{Narrator}\\
Wrzesień.

\textbf{C}\\
Czternasty. Jom Kipur.\\
Dziewiętnasty. Sukot.

% --> LISTOPAD
\textbf{Narrator}\\
Listopad.

\textbf{B}\\
Pierwszy. Wszystkich świętych.

\textbf{C}\\
Dwudziesty ósmy. Chanuka.

% --> GRUDZIEŃ
\textbf{Narrator}\\
Grudzień.

\textbf{B}\\
Dwudziesty piąty. Boże Narodzenie.



\textbf{Narrator}\\
Pszczyna, Warszawa. 1916. Sprytny Numer Jeden. 

Jest tak: wojna trwa już drugi rok, Prusy i Austro-Węgry wyrżnęły
większość swoich żołnierzy i cierpią na dramatyczny brak rekruta.
Cierpią na tyle, że oferują Polakom własne państwo --- byle tylko ci
tłumnie ruszyli na Rosję, byle tylko zasilili ich armie. Polacy pomysł
kupili, szybko zorganizowali szczupłą państwowość, a~rekruci... no
cóż, nie tym razem.

Kiedy miesiąc później w Warszawie podejmowano decyzję o zorganizowaniu
pierwszych w Królestwie Polskim wyborów samorządowych, Sprytny Numer
Jeden siedział smutny w ławie Tymczasowej Rady Stanu i ubolewał nad
losem polskiej prowincji. ,,Co będzie, jeśli Żydzi przejmą rady
gminne?'' --- rozpaczał.  ,,Czy możemy pozwolić sobie na takie ryzyko
po stu latach niewoli?''. Pewnie nie.

Sprytny Numer Jeden zadręczał się tymi pytaniami przez wiele dni, aż
wymyślił.  Strategia, którą opracował miała być bezbłędna
i zagwarantować polskość polskiej prowincji. 

\hangindent2cm
\hangafter=0
\emph{Obywatele! Od 1917 roku mamy istotny samorząd miejski oparty na
wyborach. Od naszej własnej dojrzałości zależy abyśmy wybrali Radę
Miejską, możliwie najlepszą, a~przedewszystkiem \textbf{naszą,
polską}, jak polskiem jest miasto nasze i ukochany Kraj nasz ojczysty.
Kto nie spełni swojego obowiązku ten nie jest godzien nazwy polaka
i~obywatela naszego miasta!}

W Będzinie nowa ordynacja wyborcza zagwarantowała Żydom siedemdziesiąt
procent miejsc w radzie jeszcze przed głosowaniem. Żydowskie partie % TODO rewrite till the end of paragraph
zaproponowały zatem polskim ugrupowaniom sojusze za gwarancję tych
siedemdziesięciu procent. Wybuchł skandal, polskie partie ogłosiły
bojkot wyborów, nie wystawiły żadnego kandydata. Po wyborach 
samorządowych Radę Miasta formują:

\textbf{A}\\
Abraham, Chil, Dawid, Gerszlik, Herman, Hirsch, Isaak, Izrael, Jacob,
Joachim, Juda, Józef, Michel, Mojżesz, Moszek, Salomon, Szalum


\textbf{Narrator}\\
Będzin. 1931. Sprytny Numer Dwa. 

Sprytny Numer Dwa siedział kiedyś na Wzgórzu Zamkowym i~przyglądał się
okolicznym wsiom. Carskie przepisy zabroniły Żydom osadzania się poza
miastem, więc to tam właśnie dostrzegł prawdziwą ostoję polskości. Tam
dostrzegł braci katolików --- we wsi szczęśliwej.  Zatęsknił.
Zapłakał.

A gdyby tak włączyć wszystkie te wsie w granice miasta?

Jedna decyzja, by z dnia na dzień Żydzi znów stali się mniejszością.
Jeden podpis, by żydowskie miasto stało się zaledwie jedną z dzielnic
polskiego miasta.  Dzielnicą żydowską, tą --- zamieszkałą przez Żydów. 

Jedna decyzja, by mogli mieć swoją reprezentację. Jeden podpis, by % TODO rewrite the whole paragraph
mogli startować w wyborach w imieniu będzińskich Żydów, by mogli
zadbać o własną sytuację i pozycję w mieście. 

Decyzję podjęto. Sarmacja Będzin kontra ŻKS Hakoah --- 1:0. Po
najbliższych wyborach samorządowych Radę Miasta formują:

\textbf{A}\\
Abraham, Dawid

\textbf{B}\\
Franciszek

\textbf{A}\\
Gerszlik, Hirsch, Isaak

\textbf{B}\\
Jan

\textbf{A}\\
Joachim

\textbf{B}\\
Kazimierz

\textbf{A}\\
Mojżesz,  Moszek

\textbf{B}\\
Paweł, Piotr

\textbf{A}\\
Salomon

\textbf{B}\\
Sławomir,  Stanisław

\textbf{A}\\
Szalum

\textbf{B}\\
Tomasz


\textbf{Narrator}\\
Jakże raduje się dusza księdza Zawadzkiego na myśl, że znów zamieszka
,,między swemi''.  Parafialne księgi zapełnią się katolikami, bracia
starsi w wierze staną się mniejszością\dots{} Dawny ład powróci do miasta.

\hangindent2cm
\hangafter=0
Ksiądz Zawadzki nie wie jeszcze, że za ratowanie Żydów z~podpalonej
przez nazistów synagogi odznaczony zostanie tytułem ,,Sprawiedliwego
wśród narodów świata''.

Przyłączenie okolicznych wsi do Będzina jest niepodważalne \mbox{---}~mamy
jedną parafię, jeden targ, robimy w  tych samych kopalniach. Kto by
zauważył, że to pomysł Sprytnego, a nie dziejowa konieczność
rozwijającego się miasta?  Kto by zauważył, że pomysł Sprytnego komuś
w~mieście odbierze poczucie \emph{bycia u siebie}?

To właśnie jest skuteczna strategia. Bez siły, bez jawnej przemocy.
Coś się zmieniło, bo taka jest kolej rzeczy.  Ciężko byłoby nawet
powiedzieć, co. Startegia jest skuteczna o tyle, o ile jej nie widać. 

Będzin. 1921--1933. Dumny.

Z inicjatywy Dumny  w centrum miasta powstanie pomnik.
11~Pułk Piechoty przelał tyle będzińskiej krwi, że teraz będzie mógł
spokojnie zmaterializować się na głównym placu miasta. W brązie.
Pomnik zaprojektuje prof. Adam Szyszko-Bohusz.

Zdjęcia z uroczystości odsłonięcia pomnika przepełnione są kwiatami.
Całość na bogato --- dwa rzędy wyprężonych piechurów, tłumnie zebrani
mieszkańcy miasta, ksiądz Zawadzki z kropidłem w dłoni.  Wszystko po
bożemu.  Jak to na uroczystościach miejskich.  Tylko ludzi w oknach
nie widać.

Jeszcze raz. Na archiwalnych zdjęciach widać wielką, wystawną
uroczystość z~tłumem gapiów, a w tle okna i balkony okolicznych
kamienic są kompletnie puste?  Przecież to są najlepsze miejsca. Kto
nie chciałby się tam znaleźć?  Kto nie chciałby stamtąd właśnie
oglądać uroczystości?

Chasydzi! Bo to właśnie Chasydzi mieszkają wokół placu 3~Maja i to
właśnie oni nie chcą patrzeć na uroczyste odsłonięcie pomnika. Bo ten
pomnik --- odsłonięty tuż przed ich oknami --- to z~gigantycznej błyskawicy
wznosząca ku niebu wieniec oliwny grecka bogini Nike ---~naga!

\hangindent2cm
\hangafter=0
Kiedy Niemcy wysadzą ją w grudniu trzydziestego dziewiątego, po
Będzinie krążyć będzie żart, że to jedyna dobra rzecz, jaką naziści
zrobili dla Żydów w tym mieście.

Po ponad ośmiu miesiącach okupacji Będzina, 22.05.1940 roku, komendant
45 rewiru policyjnego pisze do komendanta V odcinka policji 
ochronnej:

Zwracam się z prośbą o nadesłanie do miasta Będzina większej ilości
funkcjonariuszy policji, głównie cywilnych.  Żydzi się nas nie
słuchają. Łamią stawiane zakazy. Jeżdżą po aryjskiej stronie
w tramwajach i~autobusach. Nie ustępują miejsca na ulicy obywatelom
Trzeciej Rzeszy. Nie chcą nosić opaski z Gwiazdą Dawida oraz nie
przestrzegają godziny policyjnej. Próby zatrzymania często kończą się
na pobiciu funkcjonariuszy przez grupki żydowskie.  Jesteśmy bezsilni.
Prosimy o pomoc.

Niemcy chcieli być sprytni i pomysłowi. Chcieli tak organizować świat,
żeby wszystko działało jak w zegarku. W rzeczywistości jednak
niemiecki zegarek trzeba było za każdym razem nakręcić. Nazistom
mechanizm zawsze nakręcała II Pancerna. 

\hangindent2cm
\hangafter=0
\emph{Oczywiście mogłem się zabić, ale chciałem przeżyć\dots{}} 

\hangindent2cm
\hangafter=0
\emph{\dots{}żeby się zemścić i móc dać świadectwo\dots{}}

\hangindent2cm
\hangafter=0
\emph{\dots{}chociaż ci, którzy tego doświadczyli nigdy tego nie wysłowią!}

\hangindent2cm
\hangafter=0
\emph{Przeszłość należy do umarłych!}

W roku 1958 Będzin obchodzi rocznicę 600-lecia założenia miasta.
Z~tej okazji lokalne władze podejmują się renowacji dwóch historycznie
najważniejszych dla Będzina budowli --- zamku królewskiego i pałacu
Mieroszewskich. Decyzję o~hucznych obchodach rocznicy podjęli
jednogłośnie członkowie Rady Miasta:

\textbf{B}\\
Andrzej, Antoni, Bogusław, Franciszek, Ignacy, Jan, Kazimierz, Mirosław, 
Paweł, Piotr, Stanisław, Szczepan, Tomasz

\end{document}

Ksiądz Zawadzki zupełnie nie wiedział, co powiedzieć, gdy po raz pierwszy
zobaczył szkice przyniesione mu przez Dumnego. Stojąc na wielkiej
błyskawicy wieniec oliwny wznosi ku niebu Nike --- naga. 

\hangindent2cm
\hangafter=0
\textbf{Narrator}\\
{\color{light-gray} \emph{Reportersko}}\\
Sięgnijmy do Wikipedii. {\color{light-gray} \emph{Pauza}} Nike ---
bogini grecka, uosobienie zwycięstwa, częsty symbol na pomnikach.
Atrybuty: skrzydła, gałązka, wieniec. Ani słowa o cyckach. Na zdjęciu
Nike z Samotraki --- bez rąk, bez głowy, ale w~kiecce!

\textbf{Narrator}\\
Ksiądz Zawadzki spierał się z Dumnym do późnej nocy. ,,Nagość'' ---
,,Duma''. ,,Nagość'' --- ,,Cnota''. ,,Nagość'' --- ,,Zwycięstwo''.
Zwycięstwo --- 11.06.1933 uroczyście odsłonięto nagą Nike. 

{\color{red} [ \dots{}]}


\hangindent2cm
\hangafter=0
{\color{light-gray} \emph{Beethoven}}

{\color{red}
\textbf{Narrator}\\
Większość zakazów nałożonych na Żydów w Będzinie wprowadzono w ciągu
pierwszych miesięcy okupacji. Wśród nich znalazła się również 
godzina policyjna. Obowiązywała od 19.00.

\hangindent2cm
\hangafter=0
\textbf{Narrator}\\
{\color{light-gray} \emph{Reportersko}}\\
Udo Klausa --- landrat Będzina --- ostatecznie nie wytrzymał napięcia
w mieście i w grudniu 1942 roku ruszył na front. Jako
kapitan Wehrmachtu --- zwyklej poborowej armii --- chciał służyć
narodowi, jak każdy inny żołnierz w Europie. 
}

