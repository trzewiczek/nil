\documentclass[11pt,a4paper,oneside]{article}
\usepackage[polish]{babel}
\usepackage{polski}

\usepackage[utf8]{inputenc}
\usepackage{DejaVuSerifCondensed}
\usepackage[T1]{fontenc}
\usepackage{indentfirst}
\frenchspacing

\usepackage{marginnote}
% \usepackage[backgroundcolor=white, linecolor=white, bordercolor=white, textsize=scriptsize]{todonotes}
\usepackage[top=3cm, bottom=4cm, outer=5cm, inner=5cm, heightrounded]{geometry}

\usepackage{xcolor}
\definecolor{light-gray}{gray}{0.4}
\definecolor{konrad}{rgb}{0.16, 0.27, 0.46}
\definecolor{krzysiek}{rgb}{0.27, 0.46, 0.16}
\setlength{\parindent}{0em}
\setlength{\parskip}{2.0em}
\linespread{1.15}
\setlength{\emergencystretch}{3em}  % prevent overfull lines

\pagestyle{empty}

\begin{document}
\noindent
\textbf{nil. słuchowisko}\\
Wrocław, 25.02.2013

\textbf{Narrator}\\
Ksiądz Adamski --- proboszcz w będzińskiej parafii św. Trójcy.
Z~góry zamkowej, gdzie stoi jego kościół, widzi całe miasto.  Z~tej
okazji carska administracja przydziela mu zadanie. Będzie prowadził
spisy ludności Będzina.  Spisy powszechne, obejmujące katolików,
prawosławnych, Żydów\dots{} 

\hangindent3cm
\hangafter=0
{\color{konrad} \emph{Konrad: Chrup i loop (sprzęg)}}

Zwłaszcza Żydów --- już za chwilę wypełnią mu przecież osiemdziesiąt procent
parafialnych archiwów. 

\hangindent3cm
\hangafter=0
{\color{krzysiek} \emph{Krzysiek: Muzyka śliczna, z przestrzenią}}\\
{\color{konrad} \emph{Konrad: Do loopa dochodzą korniki w gitarze}}

{\color{light-gray} \emph{Z zadumą i tęsknotą}} \\
Ksiądz zamyśla się czasami patrząc na stojące obok kościoła ruiny
zamku.  Wyobraża sobie Będzin w okresie świetności, gdy Kazimierz
Wielki zostawił go murowanym, a szum rzeki zakłócał tętent końskich
kopyt i łopot rycerskich sztandarów\dots{}

\hangindent3cm
\hangafter=0
{\color{light-gray} \emph{Cięcie w muzyce}}

\hangindent3cm
\hangafter=0
\textbf{Nagranie}\\
{\color{light-gray} \emph{Reportersko}}\\
Za około sto trzydzieści lat podczas sesji rady miasta zatęskni tak również Adam
Szydłowski --- radny z ramienia Towarzystwa Przyjaciół Będzina.

\hangindent3cm
\hangafter=0
{\color{krzysiek} \emph{Krzysiek: Muzyka dynamicznego miasta}}\\
{\color{konrad} \emph{Konrad: Dźwiękowe ślady miasta}}

\textbf{Narrator}\\
Czasy się jednak zmieniły. Dziś Będzin to Jerozolima Zagłębia. 
Do miasta zjeżdżają tysiące Żydów z całego Królestwa Polskiego.
Inwestują w przemysł, zakładają organizacje, kłócą się i razem modlą.
Rada kahału właśnie zebrała środki na budowę wielkiej murowanej
synagogi. Wybudują ją u samego podnóża zamku. U~podnóża jego ruin.

\hangindent3cm
\hangafter=0
\textbf{Nagranie}\\
{\color{light-gray} \emph{Na muzyce}}\\
Przyjeżdżając tutaj, zastanawiałam się, czy dom mojego ojca został
zburzony, czy przetrwał. Nie spodziewałam się jednak czegoś pomiędzy.

Od otwarcia synagogi rozpoczyna się długi okres samotności będzińskich
proboszczy. Widzą z góry miasto i wiedzą, że nie należy już do nich.
Są katolikami. Należą do mniejszości. Z tęsknotą patrzą na ruiny zamku w nadziei,
że odzyska jeszcze swój blask i dominującą pozycję w mieście. 

\hangindent3cm
\hangafter=0
{\color{light-gray} \emph{Cięcie w muzyce}}

\hangindent3cm
\hangafter=0
\textbf{Nagranie}\\
{\color{light-gray} \emph{Reportersko}}\\
Tak się stanie w roku 1956, kiedy władze miasta odbudują zamek na podstawie
projektu architekta Zygmunta Gawlika. Od tej pory o Będzinie znów mówi się,
że to Królewski Gród.

\textbf{Narrator}\\
{\color{light-gray} \emph{Zmiana tonu. Czyta jak obwieszczenie}}\\
Kalendarz na rok 1924. Wydawnictwo Cechu Drukarskiego
,,Zagłębie''. Nakład 15 tys. egz. {\color{light-gray} \emph{Pauza}} Styczeń.

{\color{konrad}
\textbf{Konrad}\\
Pierwszy. Nowy Rok. \\
Szósty. Trzech Króli.
}

{\color{krzysiek}
\textbf{Krzysiek}\\
Dwudziesty piąty. Tu bi-Szwat. % koniec zimy, sadzenie drzew
}

\textbf{Narrator}\\
Luty.

{\color{konrad}
\textbf{Konrad}\\
Drugi. Ofiarowanie Pańskie.
}

{\color{krzysiek}
\textbf{Krzysiek}\\
Dwudziesty ósmy. Purim.
}

{\color{red} Uzupełnić pozostałe miesiące}

\textbf{Narrator}\\
Grudzień.

{\color{konrad}
\textbf{Konrad}\\
Dwudziesty piąty. Boże Narodzenie.
}

\hangindent3cm
\hangafter=0
{\color{light-gray} \emph{Kilkusekundowa pauza}}

\line(1,0){320}

\textbf{Narrator}\\
Pszczyna, Warszawa. 1916. Sprytny Numer Jeden. 

\hangindent3cm
\hangafter=0
{\color{light-gray} \emph{Wstaje Krzysiek, kłania się, nagranie perkusji odbija
ta-pum jak w telewizyjnym talkshow}}

Jest tak: 

\hangindent3cm
\hangafter=0
{\color{krzysiek} \emph{Krzysiek: Muzyka z kryminału}}\\
{\color{konrad} \emph{Konrad: po chwili dołącza z dyktafonem; stopniowo
dyktafonu coraz więcej}}

wojna trwa już drugi rok, Prusy i Austro-Węgry wyrżnęły większość
swoich żołnierzy i cierpią na dramatyczny brak rekruta. Cierpią na
tyle, że oferują Polakom własne państwo --- byle tylko ci tłumnie
ruszyli na Rosję, byle tylko zasilili ich armie. Polacy pomysł kupili,
szybko zorganizowali szczupłą państwowość, a~rekruci... no cóż, nie
tym razem.

Kiedy miesiąc później w Warszawie podejmowano decyzję o zorganizowaniu
pierwszych w Królestwie Polskim wyborów samorządowych, Sprytny Numer
Jeden siedział smutny w ławie Tymczasowej Rady Stanu i ubolewał nad
losem polskiej prowincji. ,,Co będzie, jeśli Żydzi przejmą rady
gminne?'' --- rozpaczał.  ,,Czy możemy pozwolić sobie na takie ryzyko
po stu latach niewoli?''. Pewnie nie.

Sprytny Numer Jeden zadręczał się tymi pytaniami przez wiele dni, aż
wymyślił.  Strategia, którą opracował miała być bezbłędna
i zagwarantować polskość polskiej prowincji. Wystarczy tylko podzielić
gminy na grupy zawodowe, a nie obszary zamieszkania, a Żydzi nie % TODO końcówka zdania
zdobędą większości w~radach.

\hangindent3cm
\hangafter=0
{\color{konrad} \emph{Konrad: dźwięk pauzujący muzykę}}

\hangindent3cm
\hangafter=0
\textbf{Nagranie}\\
Obywatele! Od 1917 roku mamy istotny samorząd miejski oparty na
wyborach. Od naszej własnej dojrzałości zależy abyśmy wybrali Radę
Miejską, możliwie najlepszą, a przedewszystkiem \textbf{naszą,
polską}, jak polskiem jest miasto nasze i ukochany Kraj nasz ojczysty.
Kto nie spełni swojego obowiązku ten nie jest godzien nazwy polaka
i obywatela naszego miasta!

\hangindent3cm
\hangafter=0
{\color{konrad} \emph{Konrad: dźwięk włączający muzykę}}

W Będzinie nowa ordynacja wyborcza zagwarantowała Żydom siedemdziesiąt
procent miejsc w radzie jeszcze przed głosowaniem. Żydowskie partie % TODO rewrite till the end of paragraph
zaproponowały zatem polskim ugrupowaniom sojusze za gwarancję tych
siedemdziesięciu procent. Wybuchł skandal, polskie partie ogłosiły
bojkot wyborów, nie wystawiły żadnego kandydata, a Radę Gminy uformowali:

\hangindent3cm
\hangafter=0
{\color{light-gray} \emph{Cięcie w muzyce}}

{\color{krzysiek}
\textbf{Krzysiek}\\
Abraham, Chil, Dawid, Gerszlik, Herman, Hirsch, Isaak, Izrael, Jacob,
Joachim, Juda, Józef, Michel, Mojżesz, Moszek, Salomon, Szalum
}

\hangindent3cm
\hangafter=0
{\color{light-gray} \emph{Kilkusekundowa pauza}}

\line(1,0){320}

 
\textbf{Narrator}\\
Będzin. 1931. Sprytny Numer Dwa. 

\hangindent3cm
\hangafter=0
{\color{light-gray} \emph{Wstaje Konrad kłania się, nagranie perkusji odbija
ta-pum}}

\hangindent3cm
\hangafter=0
{\color{konrad} \emph{Konrad: Ptaszki w tle}}

{\color{light-gray} \emph{Po dłuższej pauzie}}\\
Sprytny Numer Dwa siedział kiedyś na Wzgórzu Zamkowym i~przyglądał się
okolicznym wsiom. {\color{light-gray} \emph{Pauza}} Carskie przepisy
zabroniły Żydom osadzania się poza miastem, więc to tam właśnie
dostrzegł prawdziwą ostoję polskości. Tam dostrzegł braci katolików
--- we wsi szczęśliwej.  {\color{light-gray} \emph{Dłuższa pauza}}
Zatęsknił. Zapłakał.

\hangindent3cm
\hangafter=0
{\color{light-gray} \emph{Na pstryknięcie palcami cięcie w muzyce}}

A gdyby tak włączyć wszystkie te wsie w granice miasta?

\hangindent3cm
\hangafter=0
{\color{krzysiek} \emph{Krzysiek: Muzyka z kryminału}}\\
{\color{konrad} \emph{Konrad: na tekście dołącza z dyktafonem; stopniowo
dyktafonu coraz więcej}}

{\color{light-gray} \emph{Po pauzie}}\\
Jedna decyzja, by z dnia na dzień Żydzi znów stali się mniejszością.
Jeden podpis, by żydowskie miasto stało się zaledwie jedną z dzielnic
polskiego miasta.  Dzielnicą żydowską, tą --- zamieszkałą przez Żydów. 

Jedna decyzja, by mogli mieć swoją reprezentację. Jeden podpis, by % TODO rewrite the whole paragraph
mogli startować w wyborach w imieniu będzińskich Żydów, by mogli
zadbać o własną sytuację i pozycję w mieście. 

Decyzję podjęto. Sarmacja Będzin kontra ŻKS Hakoah --- 1:0. Po
najbliższych wyborach samorządowych Radę Miasta formują:

\hangindent3cm
\hangafter=0
{\color{light-gray} \emph{Cięcie w muzyce}}

{\color{krzysiek}
\textbf{Krzysiek}\\
Abraham, Dawid
}

{\color{konrad}
\textbf{Konrad}\\
Franciszek
}

{\color{krzysiek}
\textbf{Krzysiek}\\
Gerszlik, Hirsch, Isaak
}

{\color{konrad}
\textbf{Konrad}\\
Jan
}

{\color{krzysiek}
\textbf{Krzysiek}\\
Joachim
}

{\color{konrad}
\textbf{Konrad}\\
Kazimierz
}

{\color{krzysiek}
\textbf{Krzysiek}\\
Mojżesz,  Moszek
}

{\color{konrad}
\textbf{Konrad}\\
Paweł, Piotr
}

{\color{krzysiek}
\textbf{Krzysiek}\\
Salomon
}

{\color{konrad}
\textbf{Konrad}\\
Sławomir,  Stanisław
}

{\color{krzysiek}
\textbf{Krzysiek}\\
Szalum
}

{\color{konrad}
\textbf{Konrad}\\
Tomasz
}

\hangindent3cm
\hangafter=0
{\color{light-gray} \emph{Pauza}}\\
{\color{krzysiek} \emph{Krzysiek: Muzyka śliczna, z przestrzenią}}\\
{\color{konrad} \emph{Konrad: Korniki w gitarze}}

\textbf{Narrator}\\
Raduje się dusza księdza Zawadzkiego na myśl, że znów zamieszka
,,między swemi''. Parafialne księgi zapełnią się katolikami, 
bracia starsi staną się mniejszością\dots{} ład powróci do miasta.

\hangindent3cm
\hangafter=0
{\color{light-gray} \emph{Cięcie w muzyce}}

\hangindent3cm
\hangafter=0
\textbf{Nagranie}\\
{\color{light-gray} \emph{Reportersko}}\\
Ksiądz Zawadzki nie wie jeszcze, że za ratowanie Żydów z podpalonej
przez nazistów synagogi odznaczony zostanie tytułem ,,Sprawiedliwego
wśród narodów świata''.

\hangindent3cm
\hangafter=0
{\color{light-gray} \emph{Dwusekundowa pauza}}

\hangindent3cm
\hangafter=0
{\color{konrad} \emph{Konrad: Na tekście szum miksera niezauważalnie narasta do niewysokiego poziomu}}

\textbf{Narrator}\\
Przyłączenie okolicznych wsi do Będzina jest niepodważalne \mbox{---}~mamy
jedną parafię, jeden targ, robimy w  tych samych kopalniach. Kto by
zauważył, że to pomysł Sprytnego, a nie dziejowa konieczność
rozwijającego się miasta?  Kto by zauważył, że pomysł Sprytnego komuś
w mieście odbierze poczucie \emph{bycia u siebie}?

To właśnie jest skuteczna strategia. Bez siły, bez jawnej przemocy.
Coś się zmieniło, bo taka jest kolej rzeczy.  Ciężko byłoby nawet
powiedzieć, co. Startegia jest skuteczna o tyle, o ile jej nie widać. 

\hangindent3cm
\hangafter=0
{\color{konrad} \emph{Konrad: Po dwóch sekundach cięcie szumu uderza publiczność ciszą}}

\hangindent3cm
\hangafter=0
{\color{light-gray} \emph{Kilkusekundowa pauza}}


\line(1,0){320}

\textbf{Narrator}\\
Będzin. 1921--1933. Dumny. 

\hangindent3cm
\hangafter=0
{\color{light-gray} \emph{Wstaje Krzysiek kłania się, nagranie
perkusji odbija ta-pum}}

Z inicjatywy Dumnego w centrum miasta powstanie pomnik. 11~Pułk
Piechoty przelał tyle będzińskiej krwi, że teraz będzie mógł spokojnie
zmaterializować się na głównym placu miasta. W brązie. Pomnik
zaprojektuje prof. Adam Szyszko-Bohusz.

Zdjęcia z uroczystości odsłonięcia pomnika przepełnione są kwiatami.
Całość na bogato --- dwa rzędy wyprężonych piechurów, tłumnie zebrani
mieszkańcy miasta, ksiądz Zawadzki z kropidłem w dłoni.  Wszystko po
bożemu.  Jak to na uroczystościach miejskich.  Tylko ludzi w oknach
nie widać.

Jeszcze raz. Na zdjęciach widać wielką, wystawną uroczystość z~tłumem
gapiów, a w tle okna i balkony okolicznych kamienic są kompletnie
puste?  Przecież to są najlepsze miejsca. Kto nie chciałby się tam
znaleźć?  Kto nie chciałby stamtąd oglądać uroczystości?

Hasydzi! To właśnie oni mieszkają wokół placu 3 Maja i to właśnie oni
nie chcą patrzeć na uroczyste odsłonięcie pomnika. Dlaczego? Bo ten
pomnik odsłonięty przed ich oknami to z~gigantycznej błyskawicy
wznosząca ku niebu wieniec oliwny grecka bogini Nike --- naga!

\hangindent3cm
\hangafter=0
\textbf{Nagranie}\\
{\color{light-gray} \emph{Reportersko}}\\
Kiedy Niemcy wysadzą ją w grudniu trzydziestego dziewiątego po
Będzinie krążyć będzie żart, że to jedyna dobrza rzecz, jaką naziści
zrobili dla Żydów w tym mieście.

\line(1,0){320}

{\color{red}
PUSTE MIASTO
}

\line(1,0){320}

\hangindent3cm
\hangafter=0
{\color{light-gray} \emph{Pauza w muzyce}}

\textbf{Narrator}\\
Po ponad ośmiu miesiącach okupacji Będzina, 22.05.1940 roku, komendant 45
rewiru policyjnego

\hangindent3cm
\hangafter=0
{\color{light-gray} \emph{Konrad podrywa się jak do ukłonu. Zatrzymuje
się w pół gestu. Zespół spogląda na niego zniesmaczony. Gestem ręki
Konrad przeprasza za pomyłkę}}

Jeszcze raz. {\color{light-gray} \emph{Wdech}} Po ponad ośmiu
miesiącach okupacji Będzina, 22.05.1940 roku, komendant 45 rewiru
policyjnego pisze do komendanta V odcinka policji ochronnej:

{\color{konrad}
\textbf{Konrad}\\
Zwracam się z prośbą o nadesłanie do miasta Będzina większej ilości
funkcjonariuszy policji, głównie cywilnych. Żydzi się nas nie
słuchają. Łamią stawiane zakazy, jeżdżą po aryjskiej stronie
w tramwajach i autobusach, nie ustępują miejsca na ulicy obywatelom
Trzeciej Rzeszy, nie chcą nosić opaski z Gwiazdą Dawida oraz nie
przestrzegają godziny policyjnej. Próby zatrzymania często kończą się
na pobiciu funkcjonariuszy przez grupki żydowskie. Jesteśmy bezsilni.
Prosimy o pomoc.
}

\textbf{Narrator}\\
Niemcy chcieli być sprytni i pomysłowi. Chcieli tak organizować świat,
żeby wszystko działało jak w zegarku. W rzeczywistości jednak
niemiecki zegarek trzeba było za każdym razem nakręcić. Nazistom
mechanizm zawsze nakręcała II Pancerna. 

\line(1,0){320}

\hangindent3cm
\hangafter=0
{\color{light-gray} \emph{Bez pauzy, na strzał wchodzi agresywna
abstrakcyjna muzyka (,,na scenę wjeżdża II Pancerna''). Po około 20
sekundach muzyka zawiesza się w wyczekującym noisie (,,komory
gazowe'')}}

\hangindent3cm
\hangafter=0
{\color{light-gray} \emph{Na gest ręką muzyka staje się znów agresywna}}

\textbf{Narrator}\\
{\color{light-gray} \emph{Krzyczy}}\\
\emph{Oczywiście mogłem się zabić, ale chciałem przeżyć\dots{}} 

\hangindent3cm 
\hangafter=0 
{\color{light-gray} \emph{Na gest ręką
muzyka zawiesza się w noisie na około dwie sekundy. Na gest ręki
muzyka znów staje się agresywna}}

\emph{\dots{}żeby się zemścić i móc dać świadectwo\dots{}}

\hangindent3cm 
\hangafter=0 
{\color{light-gray} \emph{Na gest ręką
muzyka zawiesza się w noisie na około dwie sekundy. Na gest ręki
muzyka znów staje się agresywna}}

\emph{\dots{}chociaż ci, którzy tego doświadczyli nigdy tego nie wysłowią!}

\hangindent3cm 
\hangafter=0 
{\color{light-gray} \emph{Na gest ręką
muzyka zawiesza się w noisie na około dwie sekundy. Na gest ręki
muzyka znów staje się agresywna}}

\emph{Przeszłość należy do umarłych!}

\line(1,0){320}

\hangindent3cm
\hangafter=0
{\color{light-gray} \emph{Muzyka zmienia się na raz na nagranie
zespołu Mazowsze, słychać owacje, fragmenty przemówienia, okrzyki
entuzjazmu.}}

\textbf{Narrator}\\
W roku 1958 Będzin obchodzi rocznicę 600-lecia założenia miasta.
Z tej okazji lokalne władze podejmują się renowacji dwóch historycznie
najważniejszych dla Będzina budowli --- zamku królewskiego i pałacu
Mieroszewskich. Decyzję o~hucznych obchodach rocznicy podjęli
jednogłośnie radni:

\hangindent3cm
\hangafter=0
{\color{light-gray} \emph{Cięcie w muzyce}}

{\color{konrad}
\textbf{Konrad}\\
Andrzej, Antoni, Bogusław, Franciszek, Ignacy, Jan, Kazimierz, Mirosław, 
Paweł, Piotr, Stanisław, Szczepan, Tomasz
}

\end{document}

%Ksiądz Zawadzki zupełnie nie wiedział, co powiedzieć, gdy po raz pierwszy
%zobaczył szkice przyniesione mu przez Dumnego. Stojąc na wielkiej
%błyskawicy wieniec oliwny wznosi ku niebu Nike --- naga. 
%
%\hangindent3cm
%\hangafter=0
%\textbf{Nagranie}\\
%{\color{light-gray} \emph{Reportersko}}\\
%Sięgnijmy do Wikipedii. {\color{light-gray} \emph{Pauza}} Nike ---
%bogini grecka, uosobienie zwycięstwa, częsty symbol na pomnikach.
%Atrybuty: skrzydła, gałązka, wieniec. Ani słowa o cyckach. Na zdjęciu
%Nike z Samotraki --- bez rąk, bez głowy, ale w~kiecce!
%
%\textbf{Narrator}\\
%Ksiądz Zawadzki spierał się z Dumnym do późnej nocy. ,,Nagość'' ---
%,,Duma''. ,,Nagość'' --- ,,Cnota''. ,,Nagość'' --- ,,Zwycięstwo''.
%Zwycięstwo --- 11.06.1933 uroczyście odsłonięto nagą Nike. 

{\color{red} [ \dots{}]}

