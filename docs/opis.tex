\documentclass[11pt,a4paper,oneside]{article}
\usepackage[polish]{babel}
\usepackage{polski}

\usepackage[utf8]{inputenc}
\usepackage{DejaVuSerifCondensed}
\usepackage[T1]{fontenc}
\usepackage{indentfirst}
\frenchspacing

\setlength{\parindent}{0em}
\setlength{\emergencystretch}{3em}  % prevent overfull lines

\pagestyle{empty}

\begin{document}

\textbf{nil. słuchowisko}
\\

Będzin. Niewielkie miasto na zachodnim krańcu Małopolski, tuż przed
granicą zaboru. Miasto żydowskie. Mniejszość chrześcijańska to zaledwie
dwadzieścia procent mieszkańców, w skład Rady Miasta wchodzą wyłącznie
Żydzi. Ulice tętnią życiem, rozwija się przemysł, politycy łączą się w
partie polityczne, a młodzież w niezliczone organizacje. Rozkwita
kultura jidysz.

\begin{center}\rule{3in}{0.4pt}\end{center}

Będzin. Niewielkie miasto na wschodzie aglomeracji śląskiej, tuż przed
granicą województwa. Miasto polskie. Żydowska większość została
wymordowana w czasie okupacji, a powojenne władze skutecznie zatarły
resztę śladów żydowskiej obecności. Polską historię miasta tworzą
opowieści o królewskim grodzie położonym u podnóża średniowiecznego
zamku.
\\

\vspace{2em}
Nie tylko Niemcy przyczynili się do spolszczenia Będzina. Już wiele lat
przed wojną zmierzające do tego działania i decyzje przybierają postać
precyzyjnych gestów politycznych. Ten sam rodzaj działań władze PRL
podejmą po wyzwoleniu w latach 1945-1968. Jednak te wcześniejsze są
znacznie bardziej skomplikowane, toczą się bowiem w obecności większości
żydowskiej w mieście.

\hspace{1em}Słuchowisko jest próbą prześledzenia strategii, jakie polski obóz
polityczny obiera w celu wytworzenia białego, polskiego Będzina. Od
obchodzonych w 1958 roku uroczystości 600-lecia miasta, przez
poszerzanie granic powiatu w celu manipulacji statystykami wyborczymi,
po regulowanie nurtu Czarnej Przemszy i planowanie ciągów
komunikacyjnych miasta. Próbujemy krok po kroku pokazać, jak powstała
biała, współczesna Polska, jak konsekwentnie stapiano pojęcia
etniczności, narodu i państwa.


\pagebreak
\textbf{Plany dźwiękowe}

Słuchowisko realizowane jest za pomocą czterech równoległych planów
dźwiękowych:

\begin{itemize}
\item
  \emph{narracji} stanowiącej główny nośnik treści dla słuchowiska;
\item
  \emph{dźwięków źródłowych}, czyli nagrań odtwarzanych przez narratora,
  w skład których wchodzą zarówno autentyczne nagrania osób, piosenek i
  dźwięków związanych merytorycznie z narracją, jak również nagrania
  tekstów źródłowych (np. druków lub ulotek) zrealizowane współcześnie
  specjalnie na potrzeby słuchowiska;
\item
  \emph{części woklanej} realizowanej zarówno za pomocą piosenek, jak i~
  bliższych aktorstwu partii wokalnych;
\item
  \emph{muzyki}, której duża część to noise, nagrania terenowe oraz
  muzyka konkretna.
\end{itemize}

\vspace{0.5cm}

\textbf{Wątki tematyczne}

Główna oś narracji budowana jest wokół dwóch wątków tematycznych:

\begin{itemize}
\item
  \emph{strategii deterytorializacji mniejszości}, czyli takich praktyk,
    dzięki którym dokonuje się przesunięć w przestrzeni miasta, by
    zmarginalizować udział Żydów w życiu społecznym oraz odebrać im poczucie
    bycia ,,u siebie'';
\item
  \emph{strategii polityczno-prawnych} wyznaczających granice polskości
  przez szereg regulacji życia społecznego, politycznego i gospodarczego,
  dzięki którym możliwe jest podtrzymywanie podziału na Polaków i Żydów,
  na gospodarzy i tych, którzy są w Polsce jedynie przygodnie, na prawach gości.
\end{itemize}

Punktem wyjścia dla obu wątków jest przyłączenie okolicznych wsi do terytorium
Będzina w roku 1931. Ten prosty zabieg polityczny miał swój konkretny cel:
zmianę statystyk demograficznych w obrębie miasta (okoliczne wsie zamieszkałe
były --- ze względów prawnych ---~jedynie przez ludność chrześcijańską).
W konsekwencji gest ten miał rozrzedzić odsetek mieszkańców żydowskich
w mieście, doprowadzić do zmian w podziale gminy na okręgi wyborcze, a tym
samym rozdział mandatów w~Radzie Miasta. 

\hspace{1em}Po tym zdarzeniu można również zacząć
mówić o ,,dzielnicy żydowskiej''. Wcześniej Będzin był ,,miastem żydowskim'',
jednak po przyłączeniu okolicznych wsi, stara część miasta stała się zaledwie
\emph{jedną z dzielnic} miasta. 
\\

Poza główną osią narracji, treść słuchowiska realizuje się za pomocą dwóch
pobocznych wątków tematycznych obecnych na wszystkich czterech planach
dźwiękowych. Są to: 
\begin{itemize}
\item
    skala zubożenia Polski na skutek wyniszczenia mniejszości etnicznych
    w Polsce w trakcie i po II Wojnie Światowej; 
\item
    poszukiwanie emocjonalnego ekwiwalentu doświadczanego przez mniejszość
    żydowską osamotnienia i wyobcowania (bez pokuszenia o próbę doświadczenia
    lub zrozumienia opisywanych zjawisk).  
\end{itemize}

Pierwszy z powyższych wątków przejawia się np. w różnicy rozumienia pojęcia
języka obcego przed wojną i dziś.  W Polsce przedwojennej poza językami obcymi
funkcjonowały jeszcze języki sąsiadów --- ukraiński, niemiecki, jidisz, śląski,
białoruski itp. Dziś sytuacja ta znana jest jedynie na terenach pogranicza,
a dla pozostałej części kraju każdy język niepolski jest językiem obcym ---
takim, którego uczymy się od nauczycieli, na lekcjach, z podręczników, a nie 
od dzieci naszych sąsiadów lub sprzedawcy z pobliskiego sklepu tekstylnego. 

\hspace{1em}Drugi z wymienionych wątków to próba odejścia od mówienia
bezpośrenio o przemocy wobec mniejszości żydowskiej w Polsce. Przykładem jest
tu powrót do domu po wojnie. Niejednokrotnie sytuacje te były przerażająco
wyobcowujące, bo na miejscu okazywało się, że większość Żydów w dzielnicy,
miasteczku lub wsi została zamordowana, a polscy sąsiedzi przejeli majątek
powracającej osoby lub urządzili w synagodze magazyn na mąkę. Nikt nie cieszył
się z powracającego, bo nie wiadomo było, jakie będzie miał roszczenia wobec
Polaków, którzy zdążyli już poukładać sobie życie bez Żydów. Tragedia powrotu
do domu rodzinnego, w którym nikt poza powracającym nie przeżył, to zatem nie
wszystko. Poczucie osamotnienia potęguje wyobcowanie społeczne, jakie czeka na
powracającą ze strony polskich sąsiadów. Ten wątek słuchowiska nie jest jednak przeglądem
strategii wyobcowywania Żydów, lecz próbą emocjonalnego zbliżenia się do
osamotnienia, jakie było skutkiem stosowania tych strategii. 

\end{document}
