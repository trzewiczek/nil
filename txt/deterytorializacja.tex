\documentclass[11pt,a4paper,oneside]{article}
\usepackage[polish]{babel}
\usepackage{polski}

\usepackage[utf8]{inputenc}
\usepackage{DejaVuSerifCondensed}
\usepackage[T1]{fontenc}
\usepackage{indentfirst}
\frenchspacing

\setlength{\parindent}{0em}
\setlength{\parskip}{2.5em}
\linespread{1.15}
\setlength{\emergencystretch}{3em}  % prevent overfull lines

\pagestyle{empty}

\usepackage{hyperref}

\begin{document}
\textbf{nil. słuchowisko}\\
notatki do wątku: \emph{strategie deterytorializacji}

Po nadaniu uprawnień przez Kazimierza Jagelończyka w 1453 roku, Żydzi budują
w Będzinie synagogę i kirkut. Jak w całej Europie cmentarz żydowski powstaje
\emph{za murami miasta}. W kolejnych wiekach będziński kahał obejmie bardzo
duży obszar (w osi wschód-zachód około 40 kilometrów), przez co lokalny kirkut
służyć będzie Żydom z wielu miast regionu (od Bytomia przez Mysłowice,
Sosnowiec, Dąbrowę aż po Strzemieszyce). Dziś jego teren zajmuje niewielki
skwer u zbiegu ulic Zawale i Modrzejowskiej \emph{vis-\`{a}-vis} domu
pogrzebowego Hades.

Podstawowym punktem odniesienia w mieście jest wzgórze zamkowe, na którym poza
ruinami średniowiecznej warowni znajduje się parafia pod wezwaniem św. Trójcy.
Miasto usytuowane jest po południowej stronie wzgórza, u podnóża którego
znajduje się synagoga. W XIX wieku pojawia się zapotrzebowanie na nowy kirkut,
który zostaje założony na \emph{północnym} zboczu wzgórza. Patrząc z perspektywy
miasta i synagogi, nowy kirkut powstaje po drugiej stronie wzgórza zamkowego.

W roku 1931 do miasta przyłącza się okoliczne wsie. Tereny te ze względów
prawnych zamieszkałe są wyłącznie przez ludność chrześcijańską. Zmienia się
w ten sposób procentowy udział mieszkańców żydowskich w mieście. Zamiast mówić
,,miasto zamieszkałe jest prawie w całości przez Żydów'' zaczyna się mówić
,,\emph{centrum miasta} zamieszkałe jest prawie w całości przez Żydów''.
W miejsce ,,żydowskiego miasta'' pojawia się ,,dzielnica żydowska w mieście''.

W okresie międzywojennym Żydom zakazuje się prowdzenia karczm i~punktów
sprzedaży alkoholu w określonej odległości od kopalń i hut. Wyznaczenie strefy
buforowej ma przeciwdziałać rozpijaniu polskich robotników (w tym samym czasie
obowiązuje rozporządzenie zabraniające samym Żydom pracy w tych zakładach).

19 marca 1921 na placu 3 Maja wmurowano kamień węgielny pod budowę pomnika ku
czci zasłużonego w wojnie polsko-boszewickiej 11~Pułku Piechoty, którego Będzin
był miastem garnizonowym w latach 1919-1922. Pomnik --- odsłonięty 11~czerwca
1933 --- to wielka statuła Nike stojącej na błyskawicy. Projektantem pomnika
był prof.  Adolf Szyszko-Bohusz. Wokół placu 3 Maja zamieszkują będzińscy
chasydzi.  Postać nagiej kobiety na pomniku gorszy okolicznych mieszkańców.
Niemcy niszczą pomnik trzy miesiące po wkroczeniu do miasta. Społeczność
żydowska żartuje, że to jedyna dobra rzecz jaką naziści zrobili dla będzińskich
Żydów. W roku 2006 (?) miasto funduje nowy pomnik --- tym razem jest to
abstrakcyjna walcowata bryła (bez Nike).  


Na pięć dni po wejściu wojsk niemieckich do miasta, naziści podpalają synagogę
wraz z modlącymi się kilkuset Żydami. Poza doszczętnie spaloną bożnicą ogień
trawi również starą część dzielnicy żydowskiej. Następnego dnia Niemcy na pokaz
rozstrzeliwują kilku Polaków jako sprawców podpalenia.

Naziści wprowadzają w mieście godzinę policyjną o godzinie 19:00. Miasto
wyludnia się o tej godzinie do cna, co w okresie letnim tworzyło na
mieszkańcach bardzo konfudujące wrażenie dlatego, że w lecie o tej godzinie
jest jeszcze zupełnie jasno i widok pustych ulic w ciągu dnia był niezwykle
nienaturalny. W późniejszym okresie Niemcy przesuneli godzinę policyjną na
20:00.

W roku 1941 nazistowskie władze wprowadzają kolejne ograniczenia w~poruszaniu
się po ulicach miasta. Zaczyna się od rzeki po zachodnią stronę ulicy
Małachowskiego (głównej ulicy miasta). Na koniec wprowadza się całkowity zakaz
poruszania się po ulicy Małachowskiego oraz przesiedla się Żydów zamieszkałych
w kamienicach przy tej ulicy. Wysiedlenia nie obejmują mieszkańców oficyn tych
kamienic. Żeby wyjść z domu, mieszkańcy muszą przekopać się tunelami na drugą strone
kamienic.

\newpage
W roku 1943 społeczność żydowska wysiedlona zostaje do getta na Krzemionce
--- biednej dzielnicy Będzina, w której panują bardzo złe warunki życia.

W marcu 1943 roku na boisku żydowskiego klubu sportowego Hakoach naziści
organizują selekcję osób sprawnych fizycznie, by skierować ich do pracy
w będzińskich szopach, a pozostałych mieszkańców getta odesłać do obozu
w Auschwitz.

W roku 1956 (?) na terenie boiska nieistniejącego już klubu sportowego Hakoach
powstaje zajezdnia autobusowa KZK GOP.

W roku 1956 (?) znajdujące się na trzecim kirkucie w mieście macewy zostały
wykorzystane na budowę pobliskiej rampy kolejowej. Sam kirkut zostaje zalany
betonem i zostaje włączony w teren zajezdni autobusowej.

W roku 1958 z okazji 600-lecia założenia miasta Będzin w mury położonego na
rogu starego rynku (aktualnie pl. Kazimierza Wielkiego) Domu Wójtowskiego
wmurowana zostaje tablica pamiątkowa z napisem: ,,Ku wiecznej pamiątce…''. Przez
kolejne 50 lat kamienica ulega sukcesywnej dewastacji. W roku 2008 (hucznie
obchodzone 650-lecie Będzina) kamienica ma zamurowane okna oraz częściowo
zawalony dach. Tablica się zachowała. 

Z okazji 600-lecia Będzina Pałac Gzowski (rodziny Mieroszewskich herbu
Ślepowron) zostaje wyremontowany oraz otrzymuje nowy piec kaflowy przeniesiony
tutaj z jednej z kamienic w centrum miasta --- dawnej dzielnicy żydowskiej.

Na początku lat siedemdziesiątych --- w okresie budowy Huty Katowice
w sąsiedniej Dąbrowie Górniczej --- przez Będzin przeprowadzona zostaje główna
arteria miasta. Trasa wyznaczona jest w taki sposób, że konieczne staje się
wyburzenie północnej ściany Starego Rynku oraz zmiejszenie jego rozmiaru
niemalże o połowę. Dziś Stary Rynek to skwer im.  Kazimierza Wielkiego.

Sześćdziesiąt pięć lat po wysiedleniu Żydów do getta na Kamionce przyrodnia
siostra Rutki Laskier Zahava po raz pierwszy odwiedza Będzin. Jednym z punktów
zwiedzania miasta jest przedwojenna kamienica, w~której mieszkał jej ojciec
wraz z jego polską rodziną. Jest to dwupiętrowy dom z zamurowanymi oknami oraz
trzema dużymi ceglanymi podporami wspierającymi frontową ścianę budynku.
Spoglądając na kamienicę Zahava mówi: ,,Przyjeżdżając tutaj zastanawiałam się,
czy dom mojego ojca został zburzony, czy przetrwał. Nie spodziawałam się jednak
czegoś pomiędzy''. (\emph{Rutka Laskier. Polish Anna Frank},
\href{http://www.youtube.com/watch?feature=player_detailpage&v=0e_UvEZtH_E#t=180s}{fragment
z cytatem})

W roku 2007 w piwnicy kamienicy Jakuba Wienera przy ul. Małachowskiego zostaje odkryta
synagoga ,,Mizrachi''. Cała piwniczka zawalona jest ziemią. Na ścianach i suficie
znajdują się imponujące polichromie. Na jednej ze ścian polichromie pokryte są
siatką dziur po gwoździach, na których zawieszone były półki z przetworami
Polaków zamieszkujących kamienicę po wojnie.

W roku 2008 odkryte zostają polichromie w prywatnej bożnicy rodziny Curekmanów.
Prace konserwatorskie możliwe są dlatego, że zamieszkujący synagogę po wojnie
polacy byli na tyle biedni, że pokryli polichromie zwykłą białą akrylową farbą,
która nie uszkodziła olejnych malowideł. W bożnicy wybudowano jednak dodatkowe
ścianki działowe, które miały duże pomieszczenie przystosować do potrzeb
mieszkania.

\end{document}
