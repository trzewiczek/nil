\documentclass[11pt,a4paper,oneside]{article}
\usepackage[polish]{babel}
\usepackage{polski}

\usepackage[utf8]{inputenc}
\usepackage{DejaVuSerifCondensed}
\usepackage[T1]{fontenc}
\usepackage{indentfirst}
\frenchspacing

\setlength{\parindent}{0em}
\setlength{\parskip}{2.5em}
\linespread{1.15}
\setlength{\emergencystretch}{3em}  % prevent overfull lines

\pagestyle{empty}

\begin{document}
\textbf{nil. słuchowisko}\\
notatki do wątku: \emph{strategie deterytorializacji}

Jak wszędzie w Europie kirkut powstaje za murami miasta. Będziński kahał obejmuje bardzo duży obszar (w osi wschód-zachód około 40 kilometrów) przez co kirkut służy Żydom z wielu miast regionu (od Bytomia przez Mysłowice, Sosnowiec, Dąbrowę aż po Strzemieszyce). Dziś jest to niewielki skwer przy ulicy Zawale.

Podatek dymowy to opłata nakładana na właścicieli domów, którzy zobowiązani byli płacić podatek od każdego posiadanego komina. Opłaty dla Żydow były pięciokrotnie większe niż dla chrześcijańskich mieszkańców miasta. Żydzi zaczynają prowadzić kanały odprowadzające dym z piecy w taki sposób, by ostatecznie łączyły się w jeden komin. Praktyka to zostaje przez władze polskie odkryta i niesie poważne konsekwencje dla osób ją proktykujących.

Podstawowym punktem odniesienia w mieście jest wzgórze zamkowe, na którym poza ruinami średniowiecznej warowni znajduje się parafia pod wezwaniem św. Trójcy. Miasto usytuowane jest po południowej stronie wzgórza, u podnóża którego znajduje się synagoga. W XIX wieku pojawia się zapotrzebowanie na nowy kirkut, który zostaje założony po drugiej stronie wzgórza, na jego północnym zboczu.

Do miasta przyłącza się okoliczne niezamieszkałe przez Żydów wsie. Zmienia się procentowy udział mieszkańców żydowskich w mieście. Zamiast mówić “miasto zamieszkałe jest prawie w całości przez Żydów” zaczyna się mówić “centrum miasta zamieszkałe jest prawie w całości przez Żydów”. Pojawia się pojęcie “dzielnicy żydowskiej w Będzinie”.

W okresie międzywojennym Żydom zakazuje się prowdzenia karczm i punktów sprzedaży alkoholu w oległości XXXXX metrów od kopalń i hut, żeby nie rozpijali robotników. W tym samym czasie obowiązuje rozporządzenie zabraniające samym Żydom pracy w tych zakładach pracy.

19 marca 1921 wbudowano kamień węgielny pod budowę pomnika 11 Pułku Piechoty, którego Będzin był miastem garnizonowym w latach 1919-1922, a który zasłużyl się w wojnie polsko-boszewickiej. Pomnik odsłonięto 11 czerwca 1933. Pomnik to wielka statuła Nike stojąca na błyskawicy. Projektantem pomnika był prof. Adolf Szyszko-Bohusz. Wokół placu 3 Maja zamieszkują będzińscy chasydzi. Postać nagiej kobiety na pomniku gorszy okolicznych mieszkańców. Niemcy niszczą pomnik trzy miesiące po wkroczeniu do miasta. Społeczność żydowska żartuje, że to jedyna dobra rzecz jaką naziści zrobili dla będzińskich Żydów. W roku 200X miasto funduje nowy pomnik - tym razem jest to abstrakcyjna walcowata bryła bez Nike.

Na pięć dni po wejściu wojsk niemieckich do miasta, naziści podpalają synagogę wraz z modlącymi się kilkuset Żydami. Poza doszczętnie spaloną bożnicą ogień trawi również starą część dzielnicy żydowskiej. Następnego dnia Niemcy na pokaz rozstrzeliwują kilku Polaków jako sprawców podpalenia.

Naziści wprowadzają w mieście godzinę policyjną o godzinie 19:00. Miasto wyludniało się o tej godzinie do cna, co w okresie letnim tworzyło na mieszkańcach bardzo konfudujące wrażenie dlatego, że w lecie o tej godzinie jest jeszcze zupełnie jasno i widok pustych ulic w ciągu dnia był niezwykle nienaturalny. W późniejszym okresie Niemcy przesuneli godzinę policyjną na 20:00.

W roku 194X nazistowskie władze wprowadzają kolejne ograniczenia w poruszaniu się po siatce ulic miasta. Zaczyna się od…. na koniec wprowadza się całkowity zakaz poruszania się po ulicy Małachowskiego (głównej ulicy miasta) oraz przesiedla się Żydów zamieszkałych w kamienicach przy tej ulicy. Wysiedlenia nie obejmują mieszkańców oficyn tych kamienic. Żeby wyjść z domu mieszkańcy przekopują tunele na strone ulicy XXXXX.

W roku 1942 społeczność żydowska wysiedlona zostaje do getta w XXXXX

W roku 1943 społeczność żydowska wysiedlona zostaje do getta na Krzemionce - biednej dzielnicy Będzina, w której panują bardzo złe warunki życia.

W marcu 1943 roku na boisku żydowskiego klubu sportowego XXXXX naziści organizują selekcję osób sprawnych fizycznie, by skierować ich do pracy w będzińskich szopach, a pozostałych mieszkańców getta odesłać do obozu w Auschwitz.

W roku 19XX na terenie boiska nieistniejącego już klubu sportowego XXXXX powstaje zajezdnia autobusowa KZK GOP.

W roku 19XX znajdujące się na trzecim kirkucie w mieście macewy zostały wykorzystane na budowę pobliskiej rampy kolejowej. Sam kirkut zostaje zalany betonem i zostaje włączony w teren zajezdni autobusowej.

W roku 1958 z okazji 600-lecia założenia miasta Będzin w mury położonego na rogu starego rynku (aktualnie pl. Kazimierza Wielkiego) Domu Wójtowskiego wmurowana zostaje tablica pamiątkowa z napisem: “Ku wiecznej pamiątce….”. Przez kolejne 50 lat kamienica ulega sukcesywnej dewastacji. W roku 2008 (hucznie obchodzone 650-lecie Będzina) kamienica ma zamurowane okna oraz częściowo zawalony dach.

Z okazji 600-lecia Będzina Pałac Mieroszewskich (Gzowski) zostaje wyremontowany oraz otrzymuje nowy piec kaflowy przeniesiony tutaj z kamienicy z centrum miasta - dawnej dzielnicy żydowskiej.

Na początku lat 70 w okresie budowy Huty Katowice w Dąbrowie Górniczej przez Będzin przeprowadzona zostaje główna arteria miasta. Trasa wyznaczona jest w taki sposób, że konieczne staje się wyburzenie północnej ściany Starego Rynku oraz zmiejszenie jego rozmiaru niemalże o połowę. Dziś Stary Rynek to skwer im. Kazimierza Wielkiego.

65 lat po wysiedleniu Żydów do getta na Kamionce przyrodnia siostra Rutki Laskier Zahava po raz pierwszy odwiedza Będzin. Jednym z punktów zwiedzania miasta jest przedwojenna kamienica, w której mieszkał jej ojciec wraz z jego polską rodziną. Jest to dwupiętrowy dom z zamurowanymi oknami oraz trzema dużymi ceglanymi podporami wspierającymi frontową ścianę budynku. Spoglądając na kamienicę Zahava mówi: “Przyjeżdżając tutaj zastanawiałam się, czy dom mojego ojca został zburzony, czy przetrwał. Nie spodziawałam się jednak czegoś pomiędzy”. (Rutka Laskier. Polish Anna Frank, fragment z cytatem)

W roku 2007 XXXX odkrywa w piwnicy kamienicy przy ulicy XXXXX synagogę Mizrachi. Cała piwniczka zawalona jest ziemią. Na ścianach i suficie znajdują się imponujące polichromie. Na jednej ze ścian polichromie pokryte są siatką dziur po gwoździach, na których zawieszone były półki z przetworami Polaków zamieszkujących kamienicę po wojnie.

W roku 2008 odkryte zostają polichromie w prywatnej bożnicy rodziny Curekmanów. Prace konserwatorskie możliwe są dlatego, że ludzie zamieszkujący synagogę byli na tyle biedni, że pokryli polichromie zwykłą białą akrylową farbą, która nie uszkodziła olejnych polichromi. W bożnicy wybudowano jednak dodatkowe ścianki działowe, które miały duże pomieszczenie przystosować do potrzeb mieszkania.

\end{document}
